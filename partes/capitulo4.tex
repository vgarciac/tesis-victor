\chapter{Alineación de imágenes}
\label{capitulo4}
\lhead{Capítulo 4. \emph{Alineación de imágenes}}

\section{Introducción}



\section{Revisión teórica}
\subsection{Transformaciones geométricas}

El siguiente paso en el proceso de registro, luego de establecer la correspondencia entre las imágenes, es encontrar la transformación geométrica que permite alinearlas.

Existen varias formas de describir las transformaciones geométricas, la primera es algebraica, en la cual se muestra la estructura de la matriz de transformación, y la segunda es analizando las variables que se preservan, o que se mantienen invariantes luego de aplicar la transformación. Ambas características son presentadas a continuación.

%%%%%%%%%%%%%%%%%%%%%%%%%%%%%%%%%%%%%%%%%%%%%%%%%%%%%%%%%%
\subsubsection*{Isometría}

Como se observa en \ref{matriz-isometria}

\begin{equation}
	\begin{pmatrix}
	{x'}\\{y'}\\{1}
	\end{pmatrix} = 
	\begin{bmatrix}
	{\cos \theta}&{-\sin \theta}&{tx}\\
	{\sin \theta}&{\cos \theta}&{ty}\\
	{0}&{0}&{1}
	\end{bmatrix}
	\begin{pmatrix}
	{x}\\{y}\\{1}
	\end{pmatrix}
	\label{matriz-isometria}
\end{equation}

\begin{equation}
\vec{x'}= 
\begin{bmatrix}
{\vec{R}}&{\vec{t}}\\
{\vec{0^T}}&{1}
\end{bmatrix}
\vec{x}
\label{bloque-isometria}
\end{equation}


%%%%%%%%%%%%%%%%%%%%%%%%%%%%%%%%%%%%%%%%%%%%%%%%%%%%%%%%%%
\subsubsection*{Similaridad}

\begin{equation}
\begin{pmatrix}
{x'}\\{y'}\\{1}
\end{pmatrix} = 
\begin{bmatrix}
{s\cdot \cos \theta}&{-s\cdot \sin \theta}&{tx}\\
{s\cdot \sin \theta}&{s\cdot \cos \theta}&{ty}\\
{0}&{0}&{1}
\end{bmatrix}
\begin{pmatrix}
{x}\\{y}\\{1}
\end{pmatrix}
\label{matriz-similaridad}
\end{equation}

\begin{equation}
\vec{x'}= 
\begin{bmatrix}
{s\vec{R}}&{\vec{t}}\\
{\vec{0^T}}&{1}
\end{bmatrix}
\vec{x}
\label{bloque-similaridad}
\end{equation}


%%%%%%%%%%%%%%%%%%%%%%%%%%%%%%%%%%%%%%%%%%%%%%%%%%%%%%%%%%
\subsubsection*{Afín}

\begin{equation}
\begin{pmatrix}
{x'}\\{y'}\\{1}
\end{pmatrix} = 
\begin{bmatrix}
{a_{11}}&{a_{12}}&{tx}\\
{a_{2s1}}&{a_{22}}&{ty}\\
{0}&{0}&{1}
\end{bmatrix}
\begin{pmatrix}
{x}\\{y}\\{1}
\end{pmatrix}
\label{matriz-afin}
\end{equation}

\begin{equation}
\vec{x'}= 
\begin{bmatrix}
{\vec{A}}&{\vec{t}}\\
{\vec{0^T}}&{1}
\end{bmatrix}
\vec{x}
\label{bloque-afin}
\end{equation}

%%%%%%%%%%%%%%%%%%%%%%%%%%%%%%%%%%%%%%%%%%%%%%%%%%%%%%%%%%
\subsubsection*{Perspectiva}

\begin{equation}
\begin{pmatrix}
{x'}\\{y'}\\{1}
\end{pmatrix} = 
\begin{bmatrix}
{h_{11}}&{h_{12}}&{h_{13}}\\
{h_{21}}&{h_{22}}&{h_{23}}\\
{h_{31}}&{h_{32}}&{h_{33}}
\end{bmatrix}
\begin{pmatrix}
{x}\\{y}\\{1}
\end{pmatrix}
\label{matriz-perspectiva}
\end{equation}

\begin{equation}
\vec{x'}= 
\begin{bmatrix}
{\vec{A}}&{\vec{t}}\\
{\vec{v^T}}&{v}
\end{bmatrix}
\vec{x}
\label{bloque-perspectiva}
\end{equation}


%%%%%%%%%%%%%%%%%%%%%%%%%%%%%%%%%%%%%%%%%%%%%%%%%%%%%%%%%%
\section{Generación de sub-mosaicos}

\begin{algorithm}[H] %or another one check
	\caption{Registro de imágenes}
	\SetAlgoLined
	$I_{i+1}$ $\equiv$ Imagen nueva\;
	$I_{i}$ $\equiv$ ultima imagen añadida al mosaico\;
	$V_{i}$ $\equiv$ vecinos de $I_{i}$\;
	\While{puntos emparejados $\ge$ 4}{
		Emparejar puntos de $I_{i+1}$ con $I_{i}$\;
		\If{$I_{i}$ tiene vecinos}{
			\ForEach {vecino de $I_{i}$}{
				Emparejar puntos de $I_{i+1}$ con $V_{i}$\;
			}
		}
		descartar malos emparejamientos\;
		aplicar busqueda sectorizada\;
		\If{puntos totales emparejados $\le$ 3}{
			modificar criterio para descartar\;
			\If{criterio para descartar llega al minimo}{
				terminar \tcp*{no es posible emparejar imagen}
			}
		}
	}
\end{algorithm}


\subsection{Selección de imagen de referencia}
Prueba

\subsection{Matriz de transformación promedio}

\begin{algorithm}[H] %or another one check
	\caption{Calculo de matriz de homografia promedio}
	\SetAlgoLined
	
	\While{no se alcanza el maximo de iteraciones}{
		seleccionar 4 puntos aleatorios del primer sub-mosaico\;
		seleccionar los 4 puntos correspondientes en el segundo sub-mosaico\;
		calcular el punto medio para cada par de puntos correspondientes\;
		calcular la transformacion desde los puntos del primer sub-mosaico hasta los puntos medios\;
		aplicar transformacion en el primer sub-mosaico\;
		calcular error de distorsión en el primer sub-mosaico\;
		\eIf{el error es menor que el mas bajo obtenido}{
			guardar el error como el mas bajo\;
			guardar la matriz de transformación como la mejor\;
		}{
		restaurar valores del primer sub-mosaico\;
	}
}
\end{algorithm}


\section{Corrección euclidiana}
Prueba

\section{Resultados}
Resumen

\section{Resumen}
Resumen



