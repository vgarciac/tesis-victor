\chapter{Alineación de imágenes}
\label{capitulo4}
\lhead{Capítulo 4. \emph{Alineacion de imagenes}}

\section{Introducción}

Introducción



\section{Revisión teórica}
Prueba

\subsection{Transformaciones geométricas}

\section{Generación de sub-mosaicos}

\begin{algorithm}[H] %or another one check
	\caption{Registro de imágenes}
	\SetAlgoLined
	$I_{i+1}$ $\equiv$ Imagen nueva\;
	$I_{i}$ $\equiv$ ultima imagen anadida al mosaico\;
	$V_{i}$ $\equiv$ vecinos de $I_{i}$\;
	\While{puntos emparejados $\ge$ 4}{
		Emparejar puntos de $I_{i+1}$ con $I_{i}$\;
		\If{$I_{i}$ tiene vecinos}{
			\ForEach {vecino de $I_{i}$}{
				Emparejar puntos de $I_{i+1}$ con $V_{i}$\;
			}
		}
		descartar malos emparejamientos\;
		aplicar busqueda sectorizada\;
		\If{puntos totales emparejados $\le$ 3}{
			modificar criterio para descartar\;
			\If{criterio para descartar llega al minimo}{
				terminar \tcp*{no es posible emparejar imagen}
			}
		}
	}
\end{algorithm}


\subsection{Selección de imagen de referencia}
Prueba

\subsection{Matriz de transformación promedio}

\begin{algorithm}[H] %or another one check
	\caption{Calculo de matriz de homografia promedio}
	\SetAlgoLined
	
	\While{no se alcanza el maximo de iteraciones}{
		seleccionar 4 puntos aleatorios del primer sub-mosaico\;
		seleccionar los 4 puntos correspondientes en el segundo sub-mosaico\;
		calcular el punto medio para cada par de puntos correspondientes\;
		calcular la transformacion desde los puntos del primer sub-mosaico hasta los puntos medios\;
		aplicar transformacion en el primer sub-mosaico\;
		calcular error de distorsión en el primer sub-mosaico\;
		\eIf{el error es menor que el mas bajo obtenido}{
			guardar el error como el mas bajo\;
			guardar la matriz de transformación como la mejor\;
		}{
		restaurar valores del primer sub-mosaico\;
	}
}
\end{algorithm}


\section{Corrección euclidiana}
Prueba

\section{Resultados}
Resumen

\section{Conclusiones}
Resumen



