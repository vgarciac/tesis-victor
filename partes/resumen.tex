\begin{titlepage}
    \begin{center}

        \includegraphics[scale=0.5]{usb.png} \\
        \textsc {\large UNIVERSIDAD SIMÓN BOLÍVAR} \\
        \textsc{DECANATO DE ESTUDIOS PROFESIONALES\\
        COORDINACIÓN DE INGENIERÍA ELECTRÓNICA}\\
        \textbf{SISTEMA DE GENERACIÓN DE MOSAICOS 2D PARA ROBOTS MÓVILES A PARTIR DE VIDEO MONOCULAR} \\
        PROYECTO DE GRADO \\
        PRESENTADO POR: \\
        Victor Yovanni Garcia Carmona, Carnet: 12-10738

    \end{center}
% El resumen debe ser de una sola página
\addtotoc{Resumen}
\abstract
{
    \addtocontents{toc}{\vspace{1em}}
    En las tareas de exploración para el análisis del suelo en espacios aéreos o en fondo marino, es muy común emplear sistemas de adquisición basados en captura de vídeos para su posterior análisis. Recientes desarrollos tecnológicos han permitido que los algoritmos de visión por computadora emplean cámaras como sensor principal para la reconstrucción de entornos recorridos por vehículos móviles. El presente trabajo se enfoca en el análisis e implementación de distintos algoritmos para la reconstrucción de un mosaico 2D del suelo que inspecciona un robot, a partir de la información proveniente de una cámara monocular ubicada en la parte inferior de éste. El robot en cuestión puede realizar recorridos aéreos para realizar la adquisición del vídeo, o incluso trayectorias mas desafiantes como serian las aplicaciones subacuáticas. Se describe la implementación de distintos algoritmos usando técnicas de procesamiento de imágenes y visión por computadora, que conforman un sistema automatizado de generación de mapa en dos dimensiones de la superficie recorrida por el vehículo móvil. Se realizan mejoras en la detección de puntos clave y se optimiza el cálculo de las matrices de transformación para la alineación de imágenes en el mosaico, logrando así un mapa con la menor distorsión posible. Finalmente se realizaa un análisis sobre el error de proyección de dichas imágenes en el mapa del suelo generado.
    
}

% Las palabras clave son generalmente los nombres de áreas de investigación a
% los cuales está asociado el trabajo. Generalmente son tres o cuatro.
\noindent \begin{small} \textbf{Palabras clave}: mosaico, video monocular, puntos clave, matriz de transformación. 
\end{small}
	
% Iniciar nueva página luego del resumen
\clearpage
\setstretch{1.3}

\end{titlepage}
