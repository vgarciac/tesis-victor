\chapter*{Capitulo 1}
\label{intro}
\lhead{Capitulo 1. \emph{Introducción}}
\addcontentsline{toc}{chapter}{Capitulo 1}

% Descripción del problema, de lo general hacia lo específico
% \blindtext 
\section{Antecedentes}
Mensaje de prueba
\section{Justificacion y planteamiento del problema}
Mensaje de prueba justificaion
\section{Objetivos}
\subsection{Objetivo General}
Mensaje de prueba obj general
\subsection{Objetivos Especificos}
Mensaje de prueba obj Especificos
\section{Estructura del trabajo};
Mensaje de prueba estructura del trabajo
% % Trabajos anteriores
% \blindtext
% EJEMPLO DE CITA ********
% \cite{fuente1} presenta un trabajo que \ldots
% \citeauthor{fuente2} es otro autor que \ldots
% ***********************
% Objetivo general
% \blindtext

% % Objetivos específicos
% \blinditemize

% Cachucha\footnote{Lorem Ipsum.}.

% Organización del trabajo
% Se describe brevemente qué se hace en cada capítulo
% \blindtext[4]
