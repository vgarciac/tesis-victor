
\chapter{Introduccion}
\label{capitulo1}
\lhead{Capítulo 1. \emph{Introduccion}}
% De qué va a tratar el capítulo
\section{Antecedentes}

Antecedentes

\section{Justificacion y planteamiento del problema}


Para esta tarea, es común el uso de herramientas que requieren de una gran intervencion por medio del usuario para la elaboración de un mapa.
    
Atendiendo a esta necesidad, es necesario contar con un sistema que permita realizar la reconstruccion del suelo que recorre un vehiculo con la menor interaccion posible del ser humano.


\section{Objetivos}

\subsection{Objetivo General}

Analizar e implementar un sistema automatizado que permita la reconstruccion de un mapa en dos dimensiones, del entorno recorrido por robot, aereo o submarino.

\subsection{Objetivos Específicos}

\begin{itemize}
	\item Análisis comparativo de metodos vigentes en la reconstruccion de mosaicos 2D, a partir de imagenes y videos de entrada.
	\item Implementacion de modulo de preprocesamiento y correccion de entrada.
	\item Analisis comparativo de metodos de deteccion y description de puntos clave.
	\item Implementación de módulo de alineación de imagenes mediante la deteción de puntos clave.
	\item Cuantificar el error de reproyeccion y distorsion en los modelos 2D generados.
\end{itemize}

\section{Estructura del trabajo}

Luego de presentar el planteamiento del problema y la descripción del proyecto, la presente investigacion se encuentra dividida en X capitulos, organizados de la siguiente manera:

En el \textit{\textbf{Capitulo 2}} se presenta una revisión del estado del arte, en el cual se exponen los distintos metodos para la deteccicón e identificacion de puntos de interés en una imagen. Ademas, se presentan diversos algoritmos utilizados en la actualidad para la reduccion de errores en el calculo de matrices de transformación. Para terminal, se muestran las tecnicas que se utilizan para la fusion entre las imagenes, que logran reducir errores en el color, y la calidad del mapa final.

Los resultados experimentales son mostrados en el \textit{\textbf{Capitulo X}}, en este se contemplan los modelos generados de forma automatica del suelo, tanto terrestre como submarino. Además se presenta un analisis comparativo de estos resultados sobre los que proveen herramientas no automatizads.