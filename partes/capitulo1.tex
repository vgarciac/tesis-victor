\chapter{Introduccion}
\label{capitulo1}
\lhead{Capítulo 1. \emph{Introduccion}}

% De qué va a tratar el capítulo
Resumen del capitulo 1

\section{Antecedentes}
Mensaje de prueba
\section{Justificacion y planteamiento del problema}


Al realizar tareas de exploración para el análisis de espacios aereos o de fondo marino, el tiempo y el dinero son recursos muy valiosos a tomar en cuenta. En la actualidad, con el objetivo de optimizar estos recursos se suele emplear adquisicion de datos por medio de videos para su posterior análisis. Para esta tarea, es común el uso de herramientas que requieren de una gran intervencion por medio del usuario para la elaboración de un mapa.
    
Atendiendo a esta necesidad, es necesario contar con un sistema que permita realizar la reconstruccion del suelo que recorre un vehiculo con la menor interaccion posible del ser humano. En el estado del arte se describe la evolución de algoritmos utilizados para este fin. El presente trabajo tiene como finalidad la implementación de los módulos necesarios para la reconstruccion de un mosaico 2D (Dos dimensiones) de la superficie mapeada por un robot, utilizando algoritmos de procesamiento de imagenes y vision por computadora, además del analisis de distintos algoritmos sobre el error de reproyección de imagenes en un mosaico.


\section{Objetivos}
\subsection{Objetivo General}
Mensaje de prueba obj general
\subsection{Objetivos Específicos}
Mensaje de prueba obj Especificos
\section{Estructura del trabajo};
Mensaje de prueba estructura del trabajo
