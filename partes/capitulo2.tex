\chapter{Sistemas de generación de mosaico}
\label{capitulo2}
\lhead{Capítulo 2. \emph{Sistemas de generación de mosaico}}

En este capítulo se presenta una revisión teórica del estado actual de las aplicaciones e investigaciones que se han desarrollado en el área de procesamiento de imágenes, aplicado a la construcción de mosaicos, además de una reseña histórica de la evolución de dichos métodos. Debido a que la construcción de mosaicos ha sido y sigue siendo un área de investigación muy activa, existe una gran variedad de métodos y técnicas que se han empleado para este fin. En este sentido se presenta una clasificación de estos algoritmo en base al modo de abordar los módulos principales en los sistemas de generación de mosaico.

Con esto se pretende recuperar y trascender el conocimiento acumulado en esta área de estudio, además de familiarizar al lector con los conceptos básicos, necesarios para la comprensión del presente trabajo. Luego, se presenta el modelo del sistema de generación de mosaico propuesto en base a los algoritmos y técnicas que se han utilizado en las investigaciones mas recientes de esta área. Finalmente se introduce la librería de procesamiento de imágenes que se seleccionó para la implementación de todos los módulos necesarios.

\section{Estado del arte}

La elaboración de mosaicos para la construcción de mapas del suelo, se ha desarrollado incluso antes desde la era digital de la computadoras. Desde que el proceso de registrar fotografías ha existido, se comenzaron a usar para elaborar mapas topográficos \cite{primeros-mapas}, donde imágenes adquiridas a partir de globos aerostáticos o altas colinas eran unidas manualmente. Posteriormente, producto de los avances en materia de aeronáutica, el interés por la aerofotografía se incrementó en gran medida. En este mismo sentido se utilizaban aviones para el registro de imágenes a mayores altitudes, y se cubrían mayores áreas en menor cantidad de tiempo. Pero debido a que no se alcanzaban suficiente altura, y se mantenía la necesidad de registrar grandes áreas, era requerido que los mapas se construyan mediante fotografías que se superpongan, de igual forma esta tarea se llevaba a cabo mediante técnicas manuales por medio de expertos.

La necesidad de registrar áreas aun mas grandes siguió avanzando, motivado por la llegada de los satélites que eran capaces de enviar a tierra la información que obtenían de las cámaras. Los avances tecnológicos en materia de computación, y el creciente aumento de datos para esta aplicación, promovieron el desarrollo de técnicas de procesamiento digital de imágenes para dar solución a este tipo de problemas.

Con el desarrollo de cámaras cada vez mas pequeñas y portátiles, así como también la llegada de vehículos no tripulados mas compactos ---ROV, UAV, AUV--- trajeron consigo grandes avances y nuevas técnicas por parte de centros de investigación en el área de la física, robótica y visión por computadora, que buscaron aportar soluciones para la realización automática de mosaicos, con un gran enfoque en las aplicaciones mas desafiantes como los ambientes submarinos.

Tal y como se ha mencionado el proceso de generación de mosaico involucra varios pasos principales, si bien se ilustran gráficamente en la figura \ref{imagen:mosaic-process}, los podemos definir como sigue:

\begin{itemize}
	\item \textbf{Registro:} De su termino en inglés \textit{image-registration}, consiste en establecer la correspondencia geomántica entre las imágenes que componen la misma escena. Para esto, es necesario estimar la transformación geométrica que logra alinear dichas imágenes en el mismo plano.
	
	\item \textbf{Alineación:} También llamada proyección, consiste en alinear las imágenes registradas en un sistema de referencia común, es decir, con respecto a un plano re referencia. En este caso se utiliza la transformación geométrica calculada en el paso anterior.
	
	\item \textbf{Fusión:} En este paso se busca corregir los errores fotométricos o discontinuidades presentes en el mosaico luego del proceso de alineación. Estos errores aparecen, producto de errores en la estimación de las transformaciones o a cambios en la perspectiva de los objetos observados
\end{itemize}

Si bien, se han propuesto una gran cantidad de algoritmos por parte de distintos grupos de investigación en todo el mundo, esta tarea aun sigue siendo desafiante, debido mayormente a los procesos de registro y fusión de las imágenes.

Específicamente el proceso para estimar correspondencias entre las imágenes es un problema complicado, en principio debido a la naturaleza no plana de los suelos estudiados; por otro lado la reducción de discontinuidades, o inconsistencias entre imágenes consecutivas sigue siendo una tarea desafiante. De modo que la mayoría de avances e implementaciones en esta área están encaminados en resolver estos dos problemas principales, o bien mejorar los resultados de trabajos previos.

De esta forma se propone una clasificación de los algoritmos de mosaicos basada en como estos abordan los procesos de registro y fusión, además para cada clasificación se realiza una revisión teórica de cada categoría, así como también los diferentes métodos y modificaciones que han aplicado los distintos desarrolladores. 

\begin{figure}[H]
	\centerline{
		\includegraphics[width=10cm]{registration-process}}
	\caption{Proceso básico para la generación de mosaico, poner referencia, cambiar a español}
	\label{imagen:mosaic-process}
\end{figure}

\section*{Clasificación basada en el registro de imágenes}

Este proceso es muy importante para la creación de mosaicos, y básicamente es la base para estos. Cuando se registran imágenes, primero lo que se busca es encontrar la relación, o la correspondencia entre estas, teniendo en cuanta que pudieron haber sido capturadas desde distintos puntos de vista, distintos instantes de tiempo, distinta perspectiva, o incluso distintas cámaras. Luego de encontrar las zonas o puntos correspondientes, se busca estimar una matriz de transformación geométrica que permita alinearlas todas en un sistema de referencia común. Se puede decir que el registro ha sido exitoso, si se logra estimar una matriz de transformación tal que todos los puntos correspondientes se puedan unir.

Las relaciones entre las imágenes se pueden establecer utilizando distintos métodos, ya sea, emparejando puntos coincidentes, regiones enteras, o bien usando la propiedad de correlación de fase en el dominio de la frecuencia. Estos métodos para establecer correspondencias son discutidos a continuación.

\subsection*{Algoritmos en el dominio espacial}

Los algoritmos en esta categoría utilizan la información de los píxeles para establecer la relación entre imágenes, es decir, se utiliza el valor de los píxeles (intensidad) y se trata de establecer la correspondencia de estos según la ubicación en la que se encuentran. Estos se pueden separar en dos técnicas principales: basados en área o en puntos clave.

% Basados en Area

Los algoritmos basados en área, buscan relacionar dos ventanas o regiones en dos imágenes que correspondan a la misma escena. El concepto principal consiste en mover la región de interés desde la primera imagen hacia la segunda, buscando que la diferencia entre las intensidades sea la menor posible, es decir,  se trata de estimar la mejor matriz de transformación que logre reducir la diferencia de intensidades al alinear las regiones estudiadas (citar). Algunos trabajos importantes en esta clasificación utilizaron algoritmos como NCC (citar), y el MI (citar), donde estos proporcionan una métrica de igualdad entre dos imágenes. El el primer caso, el NCC mide la similitud entre las regiones estudiadas según los valores de intensidades, mientras que el MI la mide en base a la cantidad de información que comparten estas imágenes en términos de entropía. Al emplear esta técnica se logra emparejar las imágenes a nivel de píxel. Si bien se logran buenos resultados, el proceso de iterar para optimizar los parámetros de transformación y el calculo del error para cada píxel sobre las regiones, se convierte computacionalmente costoso.

% Basados en datos de Navegación



% Basados en Características

Para reducir el tiempo de computo, se utilizan algoritmos basados en la relación de características, en los cuales se tratan de detectar puntos o regiones en distintas imágenes, que correspondan con la misma ubicación. Estas características detectadas en las imágenes, se pueden evidenciar en forma de puntos aislados, curvas continuas o regiones conectadas. Luego se puede encontrar la transformación geométrica que relaciona las características de origen con las de destino, en muchos casos resolviendo una ecuación lineal. En este caso, el proceso de registro así como también el resultado del mosaico, será tan bueno como el algoritmo de detección que se utilice.

Tal y como se mencionaron los tipos de características que se pueden extraer, podemos clasificar de forma general los algoritmos de detección. Ya sea si trabajan con características locales, como lo son aquello que detectan puntos aislados. o mas globales como los basados en detección de contornos.

\subsubsection*{Detectores locales}

Al usar este tipo de algoritmos se busca encontrar la relación entre una serie de puntos dispersos que se corresponden entre dos imágenes, donde las características locales mas comunes que se suelen detectar serian esquinas, bordes, manchas, entre otros. Posteriormente el proceso de registro se completa al estimar la transformación geométrica con dicha relación de puntos, en este caso resolviendo una ecuación lineal.

Una de las ventajas principales de esta técnica para la generación de mosaicos, es que puede trabajar con imágenes consecutivas que no posean un alto nivel de cobertura. Siempre y cuando la cantidad de puntos detectados, y correctamente emparejados entre el par de imágenes supere el mínimo necesario para la solución del sistema de ecuaciones.

Diversos algoritmos detectores de características locales se han venido desarrollando desde hace mucho tiempo, y en la actualidad estos avances han permitido que el uso de este método traiga consigo muchas ventajas sobre el resto, desde variedad de aplicaciones, robustez ante distinto tipo de escenas, y velocidad de computo (siempre en función del detector a utilizar), lo que lo convierte en uno de los mas usados para la construcción de mosaicos.

\subsubsection*{Detectores globales}

Al utilizar este tipo de detectores, se busca encontrar formas, contornos, texturas, o regiones sobresalientes que se mantengan invariantes ante cambios del punto de vista o iluminación. Al igual que con los detectores locales, aquí se busca extraer tanto la posición, como el tamaño y la orientación de estas regiones.

El resto del proceso para completar la etapa de registro se mantiene igual con este método, donde la transformación geométrica se obtiene a partir de la correspondencia entre las posiciones y orientaciones de las regiones de interés que se lograron extraer. Si bien tienen buen rendimiento ante cambios de movimiento desafiantes, su uso implica un aumento en el tiempo de computo. Entre las investigaciones mas importantes dentro de esta categoría se pueden encontrar (citar).

\subsection*{Algoritmos en el dominio frecuencial}

Ya se ha visto que los algoritmos que operan en el dominio espacial cubren la mayoría de las aplicaciones e investigaciones. Sin embargo se pueden encontrar métodos que obtienen los parámetros óptimos para la transformación a partir de cálculos en el dominio de la frecuencia. 

Estos algoritmos utilizan la propiedad de correlación de fase para lograr su objetivo. Donde para un par de imágenes que se encuentran relacionadas por una simple traslación, el funcionamiento consiste en calcular la correspondiente transformada de \textit{Fourier}, luego el espectro de la potencia cruzada entre ambas. De aquí, se asegura que la fase del espectro de la potencia cruzada corresponde con la diferencia de traslación entre las dos imágenes. Finalmente el proceso continua similarmente a los métodos anteriores, alineando las imágenes según la transformación obtenida, seguido del proceso de fusionarlas.

Tal y como se explicó este proceso para una traslación, se tienen diversos trabajos como (citar) que añaden modificaciones para permitir otro tipo de transformaciones a parte de la rotación, e incluso otros que admiten cambios en la escala (citar). Si bien, los trabajos mencionados presentaron importantes, se requiere de un buen porcentaje de cobertura entre las imágenes, presenta limitaciones en los grados de libertad de las transformaciones geométricas que se pueden estimar.


\section*{Clasificación basada en el fusión de imágenes}

Si bien el proceso de registro de imágenes es fundamental para lograr un mosaico correcto, el paso final de unir las imágenes también es de gran importancia. Teniendo en cuenta que se busca aparentar que todas las imágenes componen una sola, es vital que se logre un mapa final sin inconsistencias o discontinuidades producto de los cambios en iluminación, objetos en movimiento, entre otros.

Debido a la importancia de este paso, numerosos métodos para lidiar con este tipo de problemas se han desarrollado, lo que nos permite también clasificar los algoritmos de generación de mosaico según como aborden este problema. Entre los métodos mas utilizados encontramos: fusionar las imágenes mediante cambios suavizados o buscan la mejor linea de corte entre estas.

\subsection*{Transición suavizada}

Los algoritmos de esta categoría buscan minimizar la diferencia entre dos imágenes suavizando los bordes donde se superponen. FALTA INTRO

El método mas simple para fusionar dos imágenes, consiste en realizar una suma ponderada sobre el área de superposición entre ambas, ponderando la intensidad de cada imagen a la mitad. Al realizar esta operación se suelen tener efectos indeseados como el efecto fantasma, donde se pueden observar duplicados del mismo objeto con cierto nivel de desvanecimiento. Esto de sebe a errores en el proceso de alineación de las imágenes, diferencia en la iluminación, o incluso a objetos móviles. 

Para evitar esto, se utiliza un proceso de fusión que consiste en realizar una suma ponderada entre ambas imágenes, pero dando mayor ponderación a las regiones que se encuentren mas cerca del centro de la imagen, y menor a aquellas que se encuentren cerca el borde. Si bien, se logra reducir posibles discontinuidades entre los bordes originales, el efecto fantasma aun se puede apreciar para imágenes con fuertes problemas de alineación.

Considerando este problema, y con el objetivo de realizar una unión mas robusta, se desarrolló un esquema piramidal de fusión ponderada. El proceso consiste en obtener una imagen laplaciana para distintos tamaños de escala formando así una pirámide, al mismo tiempo se va creando para cada escala una mascara, la cual será la mascara original difuminada por el efecto del filtro gaussiano. Luego para cada nivel de la pirámide se aplica el algoritmo de fusión ponderada descrito previamente donde la mascara  difuminada pondera el valor de cada píxel. Entre los trabajos que aplican este algoritmo obteniendo resultados notables se tienen (citar), logrando reducir en gran medida el efecto duplicado en las regiones de superposición.

\subsection*{Linea de corte óptima}

En lugar de buscar reducir las posibles discontinuidades a partir de una suave transición a través del borde entre dos imágenes, en este tipo de algoritmo se busca modificar este borde. Es decir, se trata de encontrar la linea de corte en el área de superposición que logre reducir la discontinuidad de texturas entre ambas imágenes. La diferencia principal de este método sobre los anteriores, es que se toma en cuenta la información presente en la región que se desea fusionar, permitiendo que se logre remover errores producidos por el efecto paralaje, o debido a objetos móviles dentro de la escena. Por otra parte, como las regiones resultantes luego de la linea de corte no se comparten información, se pueden presentar discontinuidades producto de grandes diferencias de iluminación.


\section{Esquema propuesto}

\begin{figure}[H]
	\centerline{
		\includegraphics[width=1.21\textwidth]{esquema-general}}
		\caption{Esquema propuesto para la construcción del mosaico}
	
	\label{imagen:esquema}
\end{figure}

\section{Librería de desarrollo}

\subsection{OpenCV}