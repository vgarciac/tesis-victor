\documentclass[letterpaper, 12pt, oneside]{tesis}

% Paquetes para idioma
\usepackage[spanish]{babel}
\usepackage[utf8]{inputenc}
\usepackage[fixlanguage]{babelbib}

% Otros paquetes instalados
% Básicos
\usepackage{natbib}
\usepackage{enumerate}

% Para dibujar figuras
\usepackage{tikz}

% Para cambiar el color de las letras
\usepackage{color}

% Para incluir código (básico)
\usepackage{verbatim}
\usepackage{fancyvrb}

% Para incluir hipervínculos
\usepackage{hyperref}
\usepackage{url}
\hypersetup{urlcolor=blue, colorlinks=false}

% Para más símbolos matemáticos
\usepackage{amsmath}
\usepackage{amsthm}
\usepackage{amssymb}

% Para colocar teoremas en cajas
\usepackage{mdframed}

% Para manejas la ubicacion de las figuras
\usepackage{float}
\usepackage{wrapfig}
\usepackage{graphicx}

% Algoritmos y pseudoCodigos
\usepackage{amsmath}
\selectlanguage{spanish} 
\usepackage[ruled,vlined,linesnumbered,noresetcount,spanish,onelanguage]{algorithm2e} %for psuedo code

% Paquetes locales
% Puedes agregar paquetes locales (archivos .sty) en un subdirectorio % 'paquetes'.
% Utiliza la sintaxis \usepackage{paquetes/nombrePaquete}

% Todas las imágenes se cargan del subdirectorio 'img' por defecto.
\graphicspath{{imagenes/}}

% Sangrías de 3 espacios (3 veces el espacio de la x)
\parindent 3ex 

% Interlineado
\setlength{\baselineskip}{1.5pt}

% Interpárrafo
\setlength{\parskip}{16.5pt}

\topmargin 2cm

\renewcommand{\tablename}{Tabla}
\newcommand\listsymbolname{Acrónimos y Símbolos}

\begin{titlepage}
    \title{\vspace{-2cm} \includegraphics[width=1.2in]{./usb.png} \\[.2cm]
        \large {UNIVERSIDAD SIMÓN BOLÍVAR} \\
        \textbf{DECANATO DE ESTUDIOS PROFESIONALES \\
        COORDINACIÓN DE INGENIERIA ELECTRÓNICA}
        \vfill \large {\textbf{SISTEMA DE GENERACIÓN DE MOSAICOS 2D PARA ROBOTS MÓVILES A PARTIR DE VIDEO MONOCULAR}}  \vfill}
    \author{Por: \\
        Victor Yovanni Garcia Carmona \\[1.2cm]
        \textbf{PROYECTO DE GRADO} \\
        Presentado ante la Ilustre Universidad Simón Bolívar \\
        como requisito parcial para optar al título de \\
        Ingeniero Electrónico}
    \date{Sartenejas, Marzo de 2018}
\end{titlepage}
\clearpage
\newpage
\begin{titlepage}
    \title{\vspace{-2cm} \includegraphics[width=1.2in]{./usb.png} \\[.2cm]
        \large UNIVERSIDAD SIMÓN BOLÍVAR \\
        \textbf{DECANATO DE ESTUDIOS PROFESIONALES \\
        COORDINACIÓN DE INGENIERIA ELECTRÓNICA}
        \vfill \large {\textbf{SISTEMA DE GENERACIÓN DE MOSAICOS 2D PARA ROBOTS MÓVILES A PARTIR DE VIDEO MONOCULAR}}  \vfill}
    \author{Por: \\
        Victor Yovanni Garcia Carmona \\[1.2cm]
        Realizado con la asesoría de: \\
        José de la Cruz Cappelletto Fuentes \\
        \textbf{PROYECTO DE GRADO} \\
        Presentado ante la Ilustre Universidad Simón Bolívar \\
        como requisito parcial para optar al título de \\
        Ingeniero Electrónico}
    \date{Sartenejas, Marzo de 2018}
    
\end{titlepage}


\begin{document}
\frontmatter
\maketitle
\setstretch{1.3}

% Se incluye el acta de evaluación, verificar que se corresponda
% con el formato aceptado actualmente por el Decanato.
% Pagina del acta final
\begin{titlepage}
\begin{center}

% Upper part
\includegraphics[scale=0.5]{usb.png} \\

\textsc {\large UNIVERSIDAD SIMÓN BOLÍVAR} \\
\textsc{DECANATO DE ESTUDIOS PROFESIONALES\\
COORDINACIÓN DE INGENIERÍA ELECTRÓNICA}

\bigskip
\bigskip
\bigskip
\bigskip

% Title
\textsc{ACTA FINAL PROYECTO DE GRADO}

\bigskip
\bigskip

\textsc{\bfseries Sistema de generacion de mosaico 2D para robots móviles a partir de video monocular}

\bigskip
\bigskip
\bigskip

\begin{minipage}{\textwidth}
\centering
Presentado por: \\
\textsc{\bfseries Victor Garcia} \\

\bigskip
\bigskip

Este Proyecto de Grado ha sido aprobado por el siguiente jurado examinador: \\

\bigskip
\bigskip

% Despues de cada line coloca el (los) nombre(s) de
% cada uno de los integrantes del jurado.
\line(1,0){200} \\
Jose Cappelletto\\

\bigskip
\bigskip

\line(1,0){200} \\
Nobel Certad\\


\bigskip
\bigskip

\line(1,0){200} \\
Gerardo Fernandez\\

\bigskip
\bigskip

% \line(1,0){200} \\
% @jurado3\\
\end{minipage}

\bigskip
\bigskip
\vfill

% Date/Fecha
{\large \bfseries Sartenejas, @día de Marzo de 2018}

\end{center}
\end{titlepage}
 
\begin{titlepage}
    \begin{center}

        \includegraphics[scale=0.5]{usb.png} \\
        \textsc {\large UNIVERSIDAD SIMÓN BOLÍVAR} \\
        \textsc{DECANATO DE ESTUDIOS PROFESIONALES\\
        COORDINACIÓN DE INGENIERÍA ELECTRÓNICA}\\
        \textbf{SISTEMA DE GENERACIÓN DE MOSAICOS 2D PARA ROBOTS MÓVILES A PARTIR DE VIDEO MONOCULAR} \\
        PROYECTO DE GRADO \\
        PRESENTADO POR: \\
        Victor Yovanni Garcia Carmona, Carnet: 12-10738

    \end{center}
% El resumen debe ser de una sola página
\addtotoc{Resumen}
\abstract
{
    \addtocontents{toc}{\vspace{1em}}
    Al realizar tareas de exploración para el análisis del suelo desde espacios aereos o del fondo marino, es muy comun emplear sistemas de adquisición basados en captura de videos para su posterior análisis. En la actualidad, el incremento de la tecnologia sobre el procesamiento de datos, ha permitido que los algoritmos de vision por computadora coloquen a la camara como principal sensor para la reconstruccion de entornos recorridos por vehiculos móviles. El presente trabajo se encuantra enfocado al analisis y la implementación de distintos algoritmos para la reconstrucción de un mosaico 2D (dos dimensiones), a partir de la información proveniente de una camara monocular ubicada en la parte inferior de un robot. El robot en cuestión puede realizar recorridos aereos para realizar la adquisición del video, o incluso trayectorias mas desafiantes como serian las aplicaciones subacuaticas. Para esto, se implementarán distintos algoritmos usando tecnicas de procesamiento de imagenes y vision por computadora, para la elaboracion de un sistema automatizado que permita generar un mapa en dos dimensiones de la trayectoria recorrida, con la menor distorsion posible, mejorando la detección de puntos clave, y optimizando el calculo de las matrices de transformación para la alineación de imagenes en el mosaico, ademas de realizar analisis sobre el error de reproyeccion de dichas imagenes en el mapa del suelo generado.
    
}

% Las palabras clave son generalmente los nombres de áreas de investigación a
% los cuales está asociado el trabajo. Generalmente son tres o cuatro.
\noindent \begin{small} \textbf{Palabras clave}: mosaico, video monocular, puntos clave, matriz de transformación. 
\end{small}
	
% Iniciar nueva página luego del resumen
\clearpage
\setstretch{1.3}

\end{titlepage} 
	
% Agradecimientos
\acknowledgements{
	\addtocontents{toc}{\vspace{1em}}
				
}
\clearpage
	
\pagestyle{fancy}
	
% Tabla de contenidos o índice
\lhead{\emph{Índice General}}
\tableofcontents
	
% Estos índices solamente se usan si el libro contiene figuras, tablas y
% algoritmos. Si alguno de estos no se utiliza, comentar o eliminar las líneas
% pertinentes.
\lhead{\emph{Índice de Figuras}}
\listoffigures
	
\lhead{\emph{Índice de Tablas}}
\renewcommand*\listtablename{Índice de Tablas}
\listoftables
	
%\lhead{\emph{Índice de Algoritmos}}
%\renewcommand*\listalgorithmname{Índice de algoritmos}
%\listofalgorithms
	
\setstretch{1.5}
\clearpage
\lhead{\emph{Acrónimos y símbolos}}
\listofsymbols{ll}
{
				
	% Aquí van las siglas
	% \textbf{SIFT} & Transformacion de caracteristicas invariante a la escala (del inglés: \textbf{S}cale \textbf{I}nvariant \textbf{T}eature \textbf{T}ransform)\\
	% \textbf{SURF} & Transformacion de caracteristicas Robusto y acelerado (del inglés: \textbf{S}peed-\textbf{U}up \textbf{R}obust \textbf{F}eatures)\\
	% \textbf{FAST} & Caracteristicas de pruebas de segmentos aceleradas (del inglés: \textbf{F}eatures from \textbf{A}ccelerated \textbf{S}egment \textbf{T}est)\\
	% \textbf{BRIEF} & Características elementales independientes robustas binarias (del inglés: \textbf{B}inary \textbf{R}obust \textbf{I}ndependent \textbf{E}lementary) \textbf{F}eatures) \\
	% \textbf{ORB} & FAST orientado y BRIEF rotado (del inglés: \textbf{O}riented for FAST and \textbf{R}otated \textbf{B}RIEF)\\
	% &\\
	% \hline
	% &\\
				
	% Aquí van los símbolos
	% $\iff$ & doble implicación, si y sólo si\\
	% $\Rightarrow$ & implicación lógica\\
	% $[u:=v]$ & sustitución textual de $u$ por $v$
}
	
%% ----------------------------------------------------------------
% End of the pre-able, contents and lists of things
% Begin the Dedication page
	
\setstretch{1.3}  % Return the line spacing back to 1.3
	
\pagestyle{empty}  % Page style needs to be empty for this page
	
\dedicatory{
	\textbf{Dedicatoria} \bigskip
				
	A @personasImportantes, por @razonesDedicatoria.
}
	
\addtocontents{toc}{\vspace{2em}}
	
\mainmatter
\pagestyle{fancy}
	
% Se incluye el cuerpo de la tesis en este documento.
	
% \chapter*{Capitulo 1}
\label{intro}
\lhead{Capitulo 1. \emph{Introducción}}
\addcontentsline{toc}{chapter}{Capitulo 1}

% Descripción del problema, de lo general hacia lo específico
% \blindtext 
\section{Antecedentes}
Mensaje de prueba
\section{Justificacion y planteamiento del problema}
Mensaje de prueba justificaion
\section{Objetivos}
\subsection{Objetivo General}
Mensaje de prueba obj general
\subsection{Objetivos Especificos}
Mensaje de prueba obj Especificos
\section{Estructura del trabajo};
Mensaje de prueba estructura del trabajo
% % Trabajos anteriores
% \blindtext
% EJEMPLO DE CITA ********
% \cite{fuente1} presenta un trabajo que \ldots
% \citeauthor{fuente2} es otro autor que \ldots
% ***********************
% Objetivo general
% \blindtext

% % Objetivos específicos
% \blinditemize

% Cachucha\footnote{Lorem Ipsum.}.

% Organización del trabajo
% Se describe brevemente qué se hace en cada capítulo
% \blindtext[4]

	
% El número de capítulos varía. En mi libro fueron cuatro (sin contar
% introducción y conclusión).
\chapter{Introducción}
\label{capitulo1}
\lhead{Capítulo 1. \emph{Introducción}}
% De qué va a tratar el capítulo

La navegación y exploración en áreas de difícil acceso mediante el uso de robots, es una tarea que se ha venido desarrollando en el Grupo de Investigación y Desarrollo en Mecatrónica de la USB  \textit{(GIDM)} desde hace mucho tiempo. Donde una de las aplicaciones mas demandadas, es la tarea de reconstruir un mapa 2D de la superficie mapeada por los robots utilizados. En el presente capitulo se pretende introducir los trabajos previos y avances que se han tenido en el desarrollo de este tipo aplicaciones, específicamente en el \textit{GIDM}, que dieron origen y motivación para la realización del proyecto. Además de postular un serie de problemas que el presente trabajo busca solucionar.

\section{Antecedentes}

En el \textit{GIDM} se han realizado grandes avances en el desarrollo de equipos y plataformas robóticas para actividades de investigación, exploración e inspección de ambientes no estructurados. Usualmente cuando se opera en este tipo de ambientes, en busca de realizar exploraciones mas eficientes y a mayor escala, se emplean vehículos operados remotamente \textit{ROV} (del inglés: Remotely Operated Vehicles) equipados con cámaras de vídeo. O bien, para el caso de aplicaciones subacuáticas también se suelen utilizar vehículos autónomos submarinos \textit{AUV} (del inglés: Automated Underwater Vehicles), mientras que para exploraciones aéreas se hace uso de vehículos aéreos no tripulados UAV (del ingles: Unmanned Aerial Vehicle).

En este sentido, se tienen proyectos como el presentado por \textit{Danilo, D.} \cite{danilo}, cuyo trabajo de grado consistió en el desarrollo en un sistema de operación remota para un robot submarino (\textit{ROV}), con la finalidad de implementarlo en tareas de exploración. Con objetivos similares, \textit{Said, A.} \cite{said} basó su proyecto de grado en la instrumentación y control de un robot cuadricóptero volador(\textit{AUV}). 

Adicional a los proyectos antes mencionados, en el el \textit{GIDM}  se cuenta con un vehículo submarino llamado OpenROV\footnote{ \url{https://www.openrov.com/products/openrov28/}}, el cual es un robot maniobrado remotamente de baja envergadura, diseñado especialmente para operar bajo el agua.

\begin{figure}[H]
	\centering
	\includegraphics[width=0.7\textwidth]{openrov}
	\caption{Robot móvil OpenROV}
	\label{imagen:openrov}
\end{figure}
Si bien se cuenta con un conjunto de plataformas robóticas adaptadas para tareas de exploración, hasta el momento no se han desarrollados sistemas basados en visión que permitan incluirse en la etapa de navegación y mapeo de dichos vehículos, siendo la presente investigación la primera en abordar la tarea de la reconstrucción del suelo recorrido haciendo uso únicamente de una cámara de vídeo, sensor presente en todas las plataformas robóticas antes mencionados.

\section{Justificación y planteamiento del problema}


Cuando se habla de construir un mosaico 2D, se hace referencia al proceso de recortar y alinear imágenes, de tal forma que puedan ser representadas todas juntas en una sola gran imagen. Es importante considerar que las imágenes para este tipo de aplicaciones son capturadas desde diferentes ubicaciones de la cámara, a diferencia del proceso para elaborar imágenes panorámicas, en las cuales esta ubicación es una constante. Esta característica trae consigo uno de los principales problemas en la construcción de mosaicos, y se debe al efecto paralaje. Este efecto está asociado a la diferencia entre las posiciones aparentes de los objetos, según el punto desde donde se observa.

En la figura \ref{imagen:paralaje} se aprecia un ejemplo ilustrativo de esta definición, en la cual, si nos fijamos en el punto de vista \textbf{A}, se observa el triángulo a la izquierda del circulo, mientras que desde le punto \textbf{B} este orden se encuentra invertido. 

\begin{figure}[H]
	\centering
	\includegraphics[width=6.5cm]{paralaje}
	\caption[Efectos en el cambio del punto de vista]{Efectos en el cambio del punto de vista.}
	\label{imagen:paralaje}
\end{figure}

Si bien, este problema afecta en gran medida la construcción del mosaico, no es el único presente, y se intensifican en aplicaciones de mapeo submarino, en las cuales, se evidencian efectos de distorsión de los objetos, absorción y cambios en la dirección de la luz, producto de pequeñas partículas suspendidas en el agua.

\begin{wrapfigure}{r}{0.3\textwidth}
	\begin{center}
		\vspace*{-0.5in}
		\includegraphics[width=0.3\textwidth]{hugin}
	\end{center}
	\caption{Logo del software Hugin}
\end{wrapfigure}

En el \textit{GIDM} actualmente se utilizan mecanismos manuales para la elaboración de estos mapas, en especifico, se hace uso de \textit{softwares} como Hugin\footnote{\url{hugin.sourceforge.net/}}. Este es un programa de código abierto y gratuito bajo licencia GPL\footnote{ \url{http://www.gnu.org/copyleft/gpl.html}}, el cual esta dedicado a la generación de imágenes panorámicas, incluyendo funciones para el recorte, alineación y corrección de color; además de algoritmos para la optimización de parámetros en la cámara, y corrección de distorsión. Si bien este software esta diseñado para la creación de imágenes panorámicas, permite el uso de varios tipos de proyecciones cartográficas, entre estas la rectangular, proyectando las imágenes sobre un plano recto. 

Esta practica manual, además de limitar el alcance de los sistemas embebidos para el uso en navegación automática, requiere de una inversión de tiempo importante por medio del usuario en el proceso de selección y alineación de imágenes.


Atendiendo a esta necesidad, es necesario contar con un sistema que permita realizar la reconstrucción del suelo con la menor interacción posible del ser humano. Asimismo, con el fin de poder realizar operaciones de mapeo y localización simultanea \textit{SLAM} (del ingles: Simultaneous Localization and Mapping), haciendo uso de las herramientas y robots existentes en el laboratorio, se requiere contar con un sistema basado en visión, que genere de forma automática un mapa 2D de la superficie sobre la que navega o sobrevuela el vehículo remoto, y que logre lidiar de manera efectiva ante los problemas previamente planteados.


\section{Objetivos}

\subsection{Objetivo General}

Analizar e implementar un sistema automatizado que permita la reconstrucción de un mapa en dos dimensiones, del suelo recorrido por robot, aéreo o submarino, a través de la información capturada por una cámara ubicada en su parte inferior.

\subsection{Objetivos Específicos}

\begin{itemize}
	\item Análisis comparativo de métodos vigentes en la reconstrucción de mosaicos 2D, a partir de imágenes y videos de entrada.
	\item Implementación de modulo de pre-procesamiento y corrección de entrada.
	\item Análisis comparativo de métodos de detección y descripción de puntos característicos.
	\item Implementación de módulo de alineación de imágenes mediante la detección de puntos característicos.
	\item Cuantificar el error de proyección y distorsión en los modelos 2D generados.
\end{itemize}

\section{Estructura del trabajo}

Luego de presentar el planteamiento del problema y la descripción del proyecto, la presente investigación se encuentra dividida en 5 capítulos, organizados de la siguiente manera:

En el \textit{\textbf{Capitulo 2}} se presenta una revisión del estado del arte sobre los algoritmos de generación de mosaico, en el cual se exponen los trabajos recientes y avances importantes en esta área de investigación. Al mismo tiempo, se describen los módulos principales que componen este tipo de sistemas. Luego, en base a los algoritmos y técnicas estudiadas se propone un esquema para un sistemas de generación de mosaico. Para finalizar, se presenta la librería de procesamiento de imágenes que se planteó utilizar para la implementación de los algoritmos propuestos.

El \textit{\textbf{Capitulo 3}} inicia una revisión teórica en la cual se describe el funcionamiento de los algoritmos detectores, descriptores y emparejadores de características; y posteriormente se presentan resultados de pruebas comparativas entre los mas usados para este tipo de aplicaciones. 

El módulo encargado de la alineación de imágenes en el mosaico, es descrito en el \textit{\textbf{Capitulo 4}}. Al igual que el capitulo anterior, se presenta una revisión teórica de los conceptos necesarios para su implementación. Luego, se introduce el modelo de sub mosaicos, y la implementación de un conjunto de correcciones geométricas sobre este nuevo modelo. Finalmente se muestran los resultados de los algoritmos aplicados en esta sección, seguidos de sus respectivos análisis.

En el \textit{\textbf{Capitulo 5}} se describe el modulo final del sistema, en donde se explica el funcionamiento de los algoritmos que corrigen visualmente el mosaico final. De igual forma se muestran los resultados de su implementación, seguidos de una conclusión final sobre estos.

Finalmente, en el \textit{\textbf{Capitulo 6}} se presentan las conclusiones finales, además de propuestas sobre recomendaciones y posibles implementaciones que pueden aportar mejoras y/o permitir la continuación del proyecto aquí planteado.
\chapter{Sistemas de generación de mosaico}
\label{capitulo2}
\lhead{Capítulo 2. \emph{Sistemas de generación de mosaico}}

En este capítulo se presenta una revisión teórica del estado actual de las aplicaciones e investigaciones que se han desarrollado en el área de procesamiento de imágenes, aplicado a la construcción de mosaicos, además de una reseña histórica de la evolución de dichos métodos. Debido a que la construcción de mosaicos ha sido y sigue siendo un área de investigación muy activa, existe una gran variedad de métodos y técnicas que se han empleado para este fin. En este sentido se presenta una clasificación de estos algoritmo en base al modo de abordar los módulos principales en los sistemas de generación de mosaico.

Con esto se pretende recuperar y trascender el conocimiento acumulado en esta área de estudio, además de familiarizar al lector con los conceptos básicos, necesarios para la comprensión del presente trabajo. Luego, se presenta el modelo del sistema de generación de mosaico propuesto en base a los algoritmos y técnicas que se han utilizado en las investigaciones mas recientes de esta área. Finalmente se introduce la librería de procesamiento de imágenes que se seleccionó para la implementación de todos los módulos necesarios.

\section{Estado del arte}

La elaboración de mosaicos para la construcción de mapas del suelo, se ha desarrollado incluso antes desde la era digital de la computadoras. Desde que el proceso de registrar fotografías ha existido, se comenzaron a usar para elaborar mapas topográficos \cite{primeros-mapas}, donde imágenes adquiridas a partir de globos aerostáticos o altas colinas eran unidas manualmente. Posteriormente, producto de los avances en materia de aeronáutica, el interés por la aerofotografía se incrementó en gran medida. En este mismo sentido se utilizaban aviones para el registro de imágenes a mayores altitudes, y se cubrían mayores áreas en menor cantidad de tiempo. Pero debido a que no se alcanzaban suficiente altura, y se mantenía la necesidad de registrar grandes áreas, era requerido que los mapas se construyan mediante fotografías que se superpongan, de igual forma esta tarea se llevaba a cabo mediante técnicas manuales por medio de expertos.

La necesidad de registrar áreas aun mas grandes siguió avanzando, motivado por la llegada de los satélites que eran capaces de enviar a tierra la información que obtenían de las cámaras. Los avances tecnológicos en materia de computación, y el creciente aumento de datos para esta aplicación, promovieron el desarrollo de técnicas de procesamiento digital de imágenes para dar solución a este tipo de problemas.

Con el desarrollo de cámaras cada vez mas pequeñas y portátiles, así como también la llegada de vehículos no tripulados mas compactos ---ROV, UAV, AUV--- trajeron consigo grandes avances y nuevas técnicas por parte de centros de investigación en el área de la física, robótica y visión por computadora, que buscaron aportar soluciones para la realización automática de mosaicos, con un gran enfoque en las aplicaciones mas desafiantes como los ambientes submarinos.

Tal y como se ha mencionado el proceso de generación de mosaico involucra varios pasos principales, si bien se ilustran gráficamente en la figura \ref{imagen:mosaic-process}, los podemos definir como sigue:

\begin{itemize}
	\item \textbf{Registro:} De su termino en inglés \textit{image-registration}, consiste en establecer la correspondencia geomántica entre las imágenes que componen la misma escena. Para esto, es necesario estimar la transformación geométrica que logra alinear dichas imágenes en el mismo plano.
	
	\item \textbf{Alineación:} También llamada proyección, consiste en alinear las imágenes registradas en un sistema de referencia común, es decir, con respecto a un plano re referencia. En este caso se utiliza la transformación geométrica calculada en el paso anterior.
	
	\item \textbf{Fusión:} En este paso se busca corregir los errores fotométricos o discontinuidades presentes en el mosaico luego del proceso de alineación. Estos errores aparecen, producto de errores en la estimación de las transformaciones o a cambios en la perspectiva de los objetos observados
\end{itemize}

Si bien, se han propuesto una gran cantidad de algoritmos por parte de distintos grupos de investigación en todo el mundo, esta tarea aun sigue siendo desafiante, debido mayormente a los procesos de registro y fusión de las imágenes.

Específicamente el proceso para estimar correspondencias entre las imágenes es un problema complicado, en principio debido a la naturaleza no plana de los suelos estudiados; por otro lado la reducción de discontinuidades, o inconsistencias entre imágenes consecutivas sigue siendo una tarea desafiante. De modo que la mayoría de avances e implementaciones en esta área están encaminados en resolver estos dos problemas principales, o bien mejorar los resultados de trabajos previos.

De esta forma se propone una clasificación de los algoritmos de mosaicos basada en como estos abordan los procesos de registro y fusión, además para cada clasificación se realiza una revisión teórica de cada categoría, así como también los diferentes métodos y modificaciones que han aplicado los distintos desarrolladores. 

\begin{figure}[H]
	\centerline{
		\includegraphics[width=10cm]{registration-process}}
	\caption{Proceso básico para la generación de mosaico, poner referencia, cambiar a español}
	\label{imagen:mosaic-process}
\end{figure}

\section*{Clasificación basada en el registro de imágenes}

Este proceso es muy importante para la creación de mosaicos, y básicamente es la base para estos. Cuando se registran imágenes, primero lo que se busca es encontrar la relación, o la correspondencia entre estas, teniendo en cuanta que pudieron haber sido capturadas desde distintos puntos de vista, distintos instantes de tiempo, distinta perspectiva, o incluso distintas cámaras. Luego de encontrar las zonas o puntos correspondientes, se busca estimar una matriz de transformación geométrica que permita alinearlas todas en un sistema de referencia común. Se puede decir que el registro ha sido exitoso, si se logra estimar una matriz de transformación tal que todos los puntos correspondientes se puedan unir.

Las relaciones entre las imágenes se pueden establecer utilizando distintos métodos, ya sea, emparejando puntos coincidentes, regiones enteras, o bien usando la propiedad de correlación de fase en el dominio de la frecuencia. Estos métodos para establecer correspondencias son discutidos a continuación.

\subsection*{Algoritmos en el dominio espacial}

Los algoritmos en esta categoría utilizan la información de los píxeles para establecer la relación entre imágenes, es decir, se utiliza el valor de los píxeles (intensidad) y se trata de establecer la correspondencia de estos según la ubicación en la que se encuentran. Estos se pueden separar en dos técnicas principales: basados en área o en puntos clave.

% Basados en Area

Los algoritmos basados en área, buscan relacionar dos ventanas o regiones en dos imágenes que correspondan a la misma escena. El concepto principal consiste en mover la región de interés desde la primera imagen hacia la segunda, buscando que la diferencia entre las intensidades sea la menor posible, es decir,  se trata de estimar la mejor matriz de transformación que logre reducir la diferencia de intensidades al alinear las regiones estudiadas (citar). Algunos trabajos importantes en esta clasificación utilizaron algoritmos como NCC (citar), y el MI (citar), donde estos proporcionan una métrica de igualdad entre dos imágenes. El el primer caso, el NCC mide la similitud entre las regiones estudiadas según los valores de intensidades, mientras que el MI la mide en base a la cantidad de información que comparten estas imágenes en términos de entropía. Al emplear esta técnica se logra emparejar las imágenes a nivel de píxel. Si bien se logran buenos resultados, el proceso de iterar para optimizar los parámetros de transformación y el calculo del error para cada píxel sobre las regiones, se convierte computacionalmente costoso.

% Basados en datos de Navegación



% Basados en Características

Para reducir el tiempo de computo, se utilizan algoritmos basados en la relación de características, en los cuales se tratan de detectar puntos o regiones en distintas imágenes, que correspondan con la misma ubicación. Estas características detectadas en las imágenes, se pueden evidenciar en forma de puntos aislados, curvas continuas o regiones conectadas. Luego se puede encontrar la transformación geométrica que relaciona las características de origen con las de destino, en muchos casos resolviendo una ecuación lineal. En este caso, el proceso de registro así como también el resultado del mosaico, será tan bueno como el algoritmo de detección que se utilice.

Tal y como se mencionaron los tipos de características que se pueden extraer, podemos clasificar de forma general los algoritmos de detección. Ya sea si trabajan con características locales, como lo son aquello que detectan puntos aislados. o mas globales como los basados en detección de contornos.

\subsubsection*{Detectores locales}

Al usar este tipo de algoritmos se busca encontrar la relación entre una serie de puntos dispersos que se corresponden entre dos imágenes, donde las características locales mas comunes que se suelen detectar serian esquinas, bordes, manchas, entre otros. Posteriormente el proceso de registro se completa al estimar la transformación geométrica con dicha relación de puntos, en este caso resolviendo una ecuación lineal.

Una de las ventajas principales de esta técnica para la generación de mosaicos, es que puede trabajar con imágenes consecutivas que no posean un alto nivel de cobertura. Siempre y cuando la cantidad de puntos detectados, y correctamente emparejados entre el par de imágenes supere el mínimo necesario para la solución del sistema de ecuaciones.

Diversos algoritmos detectores de características locales se han venido desarrollando desde hace mucho tiempo, y en la actualidad estos avances han permitido que el uso de este método traiga consigo muchas ventajas sobre el resto, desde variedad de aplicaciones, robustez ante distinto tipo de escenas, y velocidad de computo (siempre en función del detector a utilizar), lo que lo convierte en uno de los mas usados para la construcción de mosaicos.

\subsubsection*{Detectores globales}

Al utilizar este tipo de detectores, se busca encontrar formas, contornos, texturas, o regiones sobresalientes que se mantengan invariantes ante cambios del punto de vista o iluminación. Al igual que con los detectores locales, aquí se busca extraer tanto la posición, como el tamaño y la orientación de estas regiones.

El resto del proceso para completar la etapa de registro se mantiene igual con este método, donde la transformación geométrica se obtiene a partir de la correspondencia entre las posiciones y orientaciones de las regiones de interés que se lograron extraer. Si bien tienen buen rendimiento ante cambios de movimiento desafiantes, su uso implica un aumento en el tiempo de computo. Entre las investigaciones mas importantes dentro de esta categoría se pueden encontrar (citar).

\subsection*{Algoritmos en el dominio frecuencial}

Ya se ha visto que los algoritmos que operan en el dominio espacial cubren la mayoría de las aplicaciones e investigaciones. Sin embargo se pueden encontrar métodos que obtienen los parámetros óptimos para la transformación a partir de cálculos en el dominio de la frecuencia. 

Estos algoritmos utilizan la propiedad de correlación de fase para lograr su objetivo. Donde para un par de imágenes que se encuentran relacionadas por una simple traslación, el funcionamiento consiste en calcular la correspondiente transformada de \textit{Fourier}, luego el espectro de la potencia cruzada entre ambas. De aquí, se asegura que la fase del espectro de la potencia cruzada corresponde con la diferencia de traslación entre las dos imágenes. Finalmente el proceso continua similarmente a los métodos anteriores, alineando las imágenes según la transformación obtenida, seguido del proceso de fusionarlas.

Tal y como se explicó este proceso para una traslación, se tienen diversos trabajos como (citar) que añaden modificaciones para permitir otro tipo de transformaciones a parte de la rotación, e incluso otros que admiten cambios en la escala (citar). Si bien, los trabajos mencionados presentaron importantes, se requiere de un buen porcentaje de cobertura entre las imágenes, presenta limitaciones en los grados de libertad de las transformaciones geométricas que se pueden estimar.


\section*{Clasificación basada en el fusión de imágenes}

Si bien el proceso de registro de imágenes es fundamental para lograr un mosaico correcto, el paso final de unir las imágenes también es de gran importancia. Teniendo en cuenta que se busca aparentar que todas las imágenes componen una sola, es vital que se logre un mapa final sin inconsistencias o discontinuidades producto de los cambios en iluminación, objetos en movimiento, entre otros.

Debido a la importancia de este paso, numerosos métodos para lidiar con este tipo de problemas se han desarrollado, lo que nos permite también clasificar los algoritmos de generación de mosaico según como aborden este problema. Entre los métodos mas utilizados encontramos: fusionar las imágenes mediante cambios suavizados o buscan la mejor linea de corte entre estas.

\subsection*{Transición suavizada}

Los algoritmos de esta categoría buscan minimizar la diferencia entre dos imágenes suavizando los bordes donde se superponen. FALTA INTRO

El método mas simple para fusionar dos imágenes, consiste en realizar una suma ponderada sobre el área de superposición entre ambas, ponderando la intensidad de cada imagen a la mitad. Al realizar esta operación se suelen tener efectos indeseados como el efecto fantasma, donde se pueden observar duplicados del mismo objeto con cierto nivel de desvanecimiento. Esto de sebe a errores en el proceso de alineación de las imágenes, diferencia en la iluminación, o incluso a objetos móviles. 

Para evitar esto, se utiliza un proceso de fusión que consiste en realizar una suma ponderada entre ambas imágenes, pero dando mayor ponderación a las regiones que se encuentren mas cerca del centro de la imagen, y menor a aquellas que se encuentren cerca el borde. Si bien, se logra reducir posibles discontinuidades entre los bordes originales, el efecto fantasma aun se puede apreciar para imágenes con fuertes problemas de alineación.

Considerando este problema, y con el objetivo de realizar una unión mas robusta, se desarrolló un esquema piramidal de fusión ponderada. El proceso consiste en obtener una imagen laplaciana para distintos tamaños de escala formando así una pirámide, al mismo tiempo se va creando para cada escala una mascara, la cual será la mascara original difuminada por el efecto del filtro gaussiano. Luego para cada nivel de la pirámide se aplica el algoritmo de fusión ponderada descrito previamente donde la mascara  difuminada pondera el valor de cada píxel. Entre los trabajos que aplican este algoritmo obteniendo resultados notables se tienen (citar), logrando reducir en gran medida el efecto duplicado en las regiones de superposición.

\subsection*{Linea de corte óptima}

En lugar de buscar reducir las posibles discontinuidades a partir de una suave transición a través del borde entre dos imágenes, en este tipo de algoritmo se busca modificar este borde. Es decir, se trata de encontrar la linea de corte en el área de superposición que logre reducir la discontinuidad de texturas entre ambas imágenes. La diferencia principal de este método sobre los anteriores, es que se toma en cuenta la información presente en la región que se desea fusionar, permitiendo que se logre remover errores producidos por el efecto paralaje, o debido a objetos móviles dentro de la escena. Por otra parte, como las regiones resultantes luego de la linea de corte no se comparten información, se pueden presentar discontinuidades producto de grandes diferencias de iluminación.


\section{Esquema propuesto}

\begin{figure}[H]
	\centerline{
		\includegraphics[width=1.21\textwidth]{esquema-general}}
		\caption{Esquema propuesto para la construcción del mosaico}
	
	\label{imagen:esquema}
\end{figure}

\section{Librería de desarrollo}

\subsection{OpenCV}
\chapter{Extracción de puntos característicos}
\label{capitulo3}
\lhead{Capítulo 3. \emph{Extracción de puntos característicos}}


\section{Introducción}

En el capítulo anterior se valuaron los distintos tipos de técnicas de establecer la relación para el registro de imágenes, y en base a estos se planteó el uso de características locales para establecer la correspondencia entre imágenes. En el presente capitulo se pretende explicar en detalle el funcionamiento de los algoritmos de detección mas usados en estas aplicaciones, así como también la importancia de los avances que cada uno introduce. Después se presentan los algoritmo de emparejamiento, así como también una explicación de su funcionamiento. Finalmente, con el objetivo de caracterizar cada uno de los detectores y emparejadores, se implementa un módulo para comparar el rendimiento de estos bajo distintas condiciones. Esto nos permitirá luego seleccionar el mejor algoritmo para cada aplicación.

\section{Revisión teórica}

\subsection{Detectores y descriptores de características}

Antes mencionar la evolución de los algoritmos de detección de puntos de interés, primero es necesario definir que son estos. Los puntos de interés, puntos clave, o \textit{"features"} (en español: características) como son comúnmente llamados, son regiones en una imagen que contienen patrones específicos, lo que hace que puedan ser fácilmente seguidos o ubicados en otra imagen. Tuytelaars y Mikolajczyk \cite{Tuytelaars} definen un punto característico local como \textit{``un patrón en la imagen que difiere de su vecindario directo''}. De esta forma, se considera que los puntos característicos deben proporcionar la posibilidad de ser identificados en diferentes imágenes con el objetivo de emparejarlos.

Para alcanzar este objetivo los detectores y extractores de puntos característicos deben cumplir con ciertas propiedades que les permita funcionar bajo distintas condiciones, en concreto se busca que estos algoritmos cumplan con las siguientes propiedades:

\begin{itemize}
	\item \textbf{Robustez:} El algoritmo debe ser capaz de detectar la misma ubicación del punto característico independientemente ante cambios en la escala, rotación, traslación, iluminación, transformaciones geométricas, artefactos de compresión y ruido.
	
	\item \textbf{Repetibilidad:} El algoritmo debe ser capaz de detectar el mismo punto característico de la misma escena bajo cambios en el punto de vista.
	
	\item \textbf{Exactitud:} El detector debe localizar el punto característico de manera precisa (misma ubicación de píxel). Especialmente para tareas de alineación de imágenes.
	
	\item \textbf{Generalidad:} El algoritmo debe ser capaz de detectar puntos que pueden ser usadas en distintas aplicaciones, es decir, que detecte varios tipos de características (esquinas, burbujas, etc.)
	
	\item \textbf{Eficiencia:} El algoritmo debe ser capaz de detectar puntos característicos en nuevas imágenes a gran velocidad, para soportar aplicaciones en tiempo real.
	
	\item \textbf{Cantidad:} El algoritmo debe detectar todos, o casi todos los puntos característicos presentes en la imagen. 
	
\end{itemize}



\begin{figure}[H]
	\centering
	\includegraphics[width=0.7\textwidth]{features}
	\caption[Caracterización de regiones en una imagen]{Caracterización de regiones en una imagen\protect\footnotemark}
	\label{imagen:features}
\end{figure}
\footnotetext{\url{https://docs.opencv.org/3.0-beta/doc/py_tutorials/py_feature2d/py_features_meaning/py_features_meaning.html}}
%https://docs.opencv.org/3.0-beta/doc/py_tutorials/py_feature2d/py_features_meaning/py_features_meaning.html

Llegados a este punto es necesario definir el funcionamiento de los algoritmos detectores, y que consideran estos como puntos característicos, basados en la definición previamente planteada. Atendiendo a la imagen \ref{imagen:features}, se puede observar que se caracterizan seis áreas de interés. Analizando estos segmentos, vemos que \textbf{\textit{A}} y \textbf{\textit{B}} corresponden con superficies planas, lo que hace que sea muy difícil identificar la ubicación exacta de estas superficies en la imagen original. Por otro lado, tenemos las regiones \textbf{\textit{C}} y \textbf{\textit{D}}, las cuales corresponden con bordes en la imagen, si bien, se puede limitar en gran medida el área de búsqueda hacia toda las regiones del mismo bordes, sigue siendo difícil acertar con la ubicación correcta. Por ultimo, analizando las regiones \textbf{\textit{E}} y \textbf{\textit{F}} tenemos que corresponden a esquinas de la imagen original, en este caso se puede identificar fácilmente la ubicación exacta de la región en la imagen.

A partir de esta idea, en la cual se consideran las esquinas como regiones fácilmente identificables en una imagen, en \textit{1988} nace el primer algoritmo de detección de puntos de interés llamado Detector de esquinas de Harris \cite{harris} (nombre original en inglés: Harris Corner Detector), y como su nombre lo indica está basado en la detección de esquinas.

Retomando el concepto planteado previamente, este detector busca la diferencia de intensidad de una región con su entorno directo, es decir, se detectará una esquina para aquellas regiones que presenten una alta variación de intensidad, al desplazar la ventana estudiada en cualquier dirección. En la figura \ref{imagen:harris-window} se puede apreciar visualmente como funciona esta ventana de búsqueda.

\begin{figure}[H]
	\centering
	\includegraphics[width=0.9\textwidth]{harris-window}
	\caption[Ventana de búsqueda para de detección de esquinas]{Ventana de búsqueda para de detección de esquinas, adaptado de\protect\footnotemark}
	\label{imagen:harris-window}
\end{figure}
\footnotetext{\url{https://dsp.stackexchange.com/questions/14338/}}
%https://dsp.stackexchange.com/questions/14338/

Cuando se trabajan con detectores de características, se desea que estos sean invariantes ante la mayor cantidad de variables posibles, tal y como se menciona en la propiedad de robustez que deben tener estos algoritmos. Si bien el detector presentado anteriormente es invariante ante la traslación y la rotación (ya que las esquinas se mantienen como esquinas si son rotadas o desplazadas), no funciona de la misma forma ante cambios de escala. Como se observa en la figura \ref{imagen:corner-scale}, una región considerada como esquina, se podría considerar plana si es ampliada.

\begin{figure}[H]
	\centering
	\includegraphics[width=0.85\textwidth]{corner-scale}
	\caption[Efecto del escalado sobre las esquinas]{Efecto del escalado sobre las esquinas	\protect\footnotemark}
	\label{imagen:corner-scale}
\end{figure}
\footnotetext{\url{https://docs.opencv.org/3.0-beta/doc/py_tutorials/py_feature2d/py_sift_intro/py_sift_intro.html}}
%https://docs.opencv.org/3.0-beta/doc/py_tutorials/py_feature2d/py_sift_intro/py_sift_intro.html

Con el fin de conseguir detectar los mismos puntos ante cambios en la escala de la imagen, Lindeberg, T. \cite{log} propone un algoritmo detector de ``manchas'' multi-escala a través de la búsqueda de un máximos en el espacio de escala, el cual se crea utilizando un operador laplaciano. El Laplaciano de Gaussianas \textit{LoG} (del ingles: Laplacian-of-Gaussian), es una combinación lineal de segundas derivadas utilizado para detectar burbujas o manchas en una imagen. El funcionamiento es el siguiente: Dada una imagen de entrada, la representación para cada escala $-s-$ de la imagen se define como la convolución de la imagen con un filtro Gaussiano con desviación estándar $s$.

Este resultado brinda una fuerte respuesta positiva para burbujas oscuras y respuestas fuertes negativas para burbujas claras, ambas de un tamaño 2$s$, donde $s$ es la escala. De esta forma las características detectadas presentan una fuerte relación entre el tamaño de las estructuras en la imagen y el grado de difusión del filtro gaussiano. Donde la desviación estándar del filtro se usa para controlar la escala cambiando que tanto se difumina la imagen.

Llegados aquí, una vez se haya detectado la ubicación de los puntos característicos en la imagen, la información de la localidad de este debe ser codificada y almacenada, y de esta forma lograr tener un descriptor único de la región con el objetivo final de ubicarlo en otra imagen. Con este fin se desarrollaron los algoritmos descriptores, los cuales una vez tengan la ubicación de los puntos característicos se encargan de convertir la información de su alrededor en una serie de números, o un vector que permita diferenciar un punto clave de otro. Esta información también es necesaria que sea invariante ante las variable mencionadas previamente, para lograr una identificación eficiente del mismo punto en distintas imágenes bajo distintas condiciones.

Partiendo de estos problemas, y del hecho que el calculo del operador LoG es computacionalmente costoso, en \textit{2004 D. Lowe} crea el detector y descriptor \textit{SIFT} \cite{sift} (del inglés: Scale Invariant Feature Transform), en el cual el espacio de escala es construido en forma piramidal con la diferencia de gaussianas DoG (del inglés: Difference of Gaussians). En este sentido, El operador DoG ofrece una aproximación al LoG, donde se calcula sin convolución restando niveles de escala adyacentes de una pirámide gaussiana. El proceso para la detección y descripción de puntos de interés de este algoritmo, consta de cuatro pasos principales:

En primer lugar, realiza una detección de máximos en el espacio de la escala aplicando la diferencia gaussiana \textit{DoG}. Para esto, se aplica el filtro gaussiano con distintos tamaños de media (se tienen distintas escalas), luego restando estas imágenes para distintos pares de escalas se logra la diferencia de gaussiana. Posteriormente se buscan los máximos locales a lo largo del espacio (coordenadas X,Y) para cada correspondiente escala. Este proceso de detección se puede visualizar en la figura \ref{imagen:sift-escalas}.

\begin{figure}[H]
	\centering
	\includegraphics[width=0.85\textwidth]{sift-escalas}
	\caption[Detector SIFT]{Detección de máximos en el espacio de la escala DoG, adaptado de \cite{sift}}
	\label{imagen:sift-escalas}
\end{figure}

En segundo lugar para la localización de puntos de interés, se descartan los puntos encontrados en el paso anterior que no superen cierto valor de umbral, es decir, que no estén lo suficientemente contrastados con su entorno. Con esta etapa el algoritmo solo toma en cuenta los puntos claves mas fuertes por cada escala. Además, con el objetivo de eliminar los bordes suficientemente contrastados que no correspondan con esquinas, el algoritmo usa una matriz hessiana para calcular las curvaturas principales, y así quedarse solo con esquinas.

Para garantizar la invarianza con respecto a la rotación, se toman los píxeles vecinos al punto clave y se calcula la magnitud y dirección del gradiente en esa región. Con esto se hace un histograma de la magnitud del gradiente en cada dirección, donde el pico mayor del histograma indica la orientación. En el caso que exista un pico mayor al 80\% del pico principal, este se utiliza para crear otro punto de interés en la misma posición pero con la distinta rotación.

Finalmente para crear el vector descriptor por cada punto clave se crea una matriz de 16x16 alrededor de este, dividida en 4 subregiones de 4x4 píxeles con un histograma de orientaciones para cada uno. Seguidamente, el descriptor del punto será el vector con los valores de los histogramas de las regiones 4x4 concatenados. La figura \ref{imagen:descriptor} la representación del descriptor de SIFT.

\begin{figure}[H]
	\centering
	\includegraphics[width=0.7\textwidth]{sift-descriptor}
	\caption[Descriptor SIFT]{Izquierda: imagen de gradientes. Derecha: descriptor del punto clave. de \cite{sift}}
	\label{imagen:descriptor}
\end{figure}


En el año 2006, un grupo de tres  personas Bay, H., Tuytelaars, T. and Van Gool, L. desarrollan \textit{SURF} \cite{surf}, el cual es un detector y descriptor de características basado en SIFT, pero con modificaciones que aumentan su velocidad de detección. Si bien, sacrifica un poco de rendimiento y precisión, lo hace mas provechoso para aplicaciones embebidas que demanden mayor velocidad de computo y menor uso de recursos, como por ejemplo \textit{SLAM}. El proceso para la extracción de características por parte de este algoritmo se compone de los siguientes pasos:

Como primer paso, en lugar de aproximar el laplaciano de Gauss \textit{LoG} (del inglés: Laplacian of Gaussians) con la diferencia de Gaussianas (DoG) como lo hace SIFT, este algoritmo aproxima LoG con cuadrados para promediar la imagen. La ventaja de aplicar filtros con cuadrados es que con la ayuda de imágenes integrales el cálculo computacional se reduce en gran medida.

\begin{figure}[H]
	\centering
	\includegraphics[width=0.8\textwidth]{surf}
	\caption[Detector y descriptor SURF]{Izquierda: aproximación a la derivada de segundo orden del filtro gaussiano (derivada parcial en el eje \textit{y}) y su aproximación con un filtro cuadrado. Derecha: vector de orientación del descriptor. Adaptado de \cite{surf}}
	\label{imagen:surf}
\end{figure}

En función de identificar la orientación, el algoritmo utiliza la respuesta wavelet Haar en horizontal y vertical en un vecindario de 6$s$ (donde $s$ es la escala evaluada) píxeles al rededor del punto de interés, Luego estas respuestas son representadas como puntos en el espacio, para luego calcular la orientación dominante con la suma de todos los resultados dentro de una ventana deslizante de apertura 60$^\circ$. En la figura \ref{imagen:surf} se puede visualizar en el lado izquierdo, la aproximación que realiza de la derivada de segundo orden del filtro gaussiano, y su aproximación con un filtro cuadrado. Del lado derecho se ilustra el vector de orientación en función a la distribución de puntos estudiados.

El siguiente avance importante en los algoritmos de detección aparece en el año 2011 con \textit{ORB} \cite{orb} (del inglés: Oriented FAST and Rotated BRIEF), este utiliza una combinación del detector FAST (del inglés: Features from Accelerated Segment Test) y del descriptor BRIEF (del inglés: Binary Robust Independent Elementary Features), este nuevo algoritmo esta caracterizado por su alta velocidad de procesamiento manteniendo un buen rendimiento, gracias al uso de un descriptor binario. 

Como se mencionó utiliza el algoritmo FAST, el cual consiste en encontrar esquinas evaluando los píxeles en un perímetro circular, de esta forma, un punto será detectado como esquina si la cantidad de píxeles de color opuesto al evaluado, supera cierto valor de umbral (ver izquierda en la figura \ref{imagen:orb}), posteriormente con el fin de aumentar la robustez, es aplicado el algoritmo de clasificación de esquinas de $Harris$. De igual forma se realiza con una estructura piramidal evaluando varias escalas (al igual que SIFT).

\begin{figure}[H]
	\centering
	\includegraphics[width=0.9\textwidth]{orb}
	\caption[Detector y descriptor ORB]{izquierda: detección de esquinas usando FAST, de \cite{fast}. derecha: descriptor basado en BRIEF, adaptado de\protect\footnotemark }
	\label{imagen:orb}
\end{figure}

\footnotetext{\url{https://gilscvblog.com/2013/10/04/}}
Como el algoritmo FAST no toma en cuenta la orientación, en el ORB se modificó para que calculara la orientación de la siguiente forma: Se considera una región ubicada en el centro del punto estudiado, luego se calcula el centroide de la región en función a la intensidad de los puntos. De esta forma, la dirección del vector desde el punto central  hasta el centroide es asignado como vector de orientación. Observando a la derecha en \ref{imagen:orb} se aprecia un ejemplo del lugar del centroide \textit{(C)} y del centro \textit{(O)} para una región en particular.


Para el descriptor utiliza BRIEF, a diferencia de los anteriores (SIFT y SURF) este es un descriptor binario y no vectorial. El descriptor BFIEF produce una palabra de $n$-bits usando el algoritmo \textit{Local Binay Tests} (LBT), el problema de esta representación es que no es muy robusta ante cambios en la rotación. Para resolver esto ORB utiliza la información de la orientación previamente calculada en el paso de detección para aplicar LBT en esa orientación.

Los algoritmos de detección que se mencionaron hasta este momento tienen una característica en común, y es que cuando trabajan con el esquema piramidal lo hacen bajo el espacio de escala Gaussiano, el cual es una instancia particular de difusión lineal. De esta forma, al utilizar este filtro no se respetan los limites naturales de los objetos y se difumina del mismo nivel toda la región de la imagen cuando se avanza entre nieveles de escala.

Enfocándose en esta característica, en el año de \textit{2012} se desarrolla el detector y descriptor llamado KAZE \cite{kaze} por parte de \textit{Pablo Fernández Alcantarilla}. Este novedoso algoritmo opera completamente en un espacio de escala no lineal, y para ello utilizan un esquema de división de operadores aditivos (\textit{AOS}, del inglés: Additive Operator Splitting), que les permite obtener espacios de escala no lineales de forma eficiente. De este modo se puede realizar un difuminado localmente adaptativo, posibilitando que se remueva el ruido en las imágenes, manteniendo información importante sobre los bordes de los objetos al avanzar en el espacio de escala. En la figura \ref{imagen:kaze} se puede observar como afecta en los bordes de los objetos el aplicar un filtro de difusión lineal, y uno que no lo es, bajo el esquema propuesto por este algoritmo.

\begin{figure}[H]
	\centering
	\includegraphics[width=0.85\textwidth]{kaze}
	\caption[Filtro no lineal propuesto por KAZE]{\textit{(A)}: imagen original, \textit{(B)} filtro lineal Gaussiano, \textit{(C)} filtro no lineal propuesto en KAZE, adaptada de \cite{kaze}}
	\label{imagen:kaze}
\end{figure}

Bajo este mismo esquema de difusión no lineal, el mismo autor en el año \textit{2013} desarrolla la versión acelerada de este algoritmo que recibe el nombre de \textit{A-KAZE} \cite{akaze} (del ingles: Accelerated KAZE). Esta mejora se utiliza un esquema basado en difusión explícita rápida \textit{FED} (del ingles: Fast Explicit Difussion) en lugar de \textit{AOS}, el cual es un nuevo esquema piramidal que incrementa en gran medida la velocidad de computo para construir el espacio de escala no lineal.

Para el calculo de la orientación el primer algoritmo \textit{KAZE} utiliza un descriptor para la orientación similar al que emplea SURF. Este encuentra la orientación dominante en un área circular de radio 6$s$ ($s$ corresponde con la escala), y para cada muestra del círculo se calcula la derivada de primer orden en las direcciones $X$ e $Y$, y se ponderan con una gaussiana centrada en el punto de interés. Luego, las respuestas de estas derivadas son representadas como puntos en un espacio vectorial, donde la orientación dominante se haya sumando las respuestas dentro de un segmento de circulo deslizante con apertura de 60$^\circ$.

Por otro lado, la versión acelerada \textit{A-KAZE} emplea un descriptor basado en una versión modificada del algoritmo de diferencia local binaria \textit{LDB} \cite{ldb} (del ingles: Local Difference Binary), llamado M-LBD (del ingles: Modified Local Difference Binary), el cual aprovecha al máximo la información del espacio de escala no lineal. La modificación consiste en hacer un sub-muestreo de cada región que divide la zona del descriptor, en lugar de calcular el promedio de todos los píxeles de la región, es decir, se tienen muestras de cada subdivisión para distintas escalas.


\subsection{Emparejadores de puntos característicos}

En este punto ya hemos estudiado los distintos algoritmos que permiten encontrar y clasificar puntos de interés en las imágenes. Ahora bien, es necesario identificar cuales de estos puntos corresponden con la misma ubicación, tal y como podemos observar en la figura \ref*{imagen:match}. Para establecer esta relación se utilizan los algoritmos emparejadores de características, los cuales relacionan estos puntos en base a los descriptores de dichos puntos.

\begin{figure}[H]
	\centering
	\includegraphics[width=14cm]{match}
	\caption[Emparejamiento de puntos característicos]{Emparejamiento de puntos que corresponden a la misma ubicación, de \cite{comp-vision} }
	\label{imagen:match}
\end{figure}

El proceso para realizar el emparejamiento consiste en el siguiente: Teniendo un punto característico $P_{1}$ perteneciente a la imagen $1$, y por otro lado se teniendo un punto $P_{2}$ perteneciente a la imagen $2$, Se calcula la distancia entre los descriptores $D_{1}$ y $D_{2}$. En el caso de descriptores vectoriales, esta distancia corresponde con la distancia \textit{Euclidiana}, dada por la siguiente expresión:
\begin{displaymath}
D = \sqrt{ (v_{1}-q_{1})^2 + (v_{2}-q_{2})^2 + \cdots + (v_{n}-q_{n})^2 }
\end{displaymath}
Donde $v_{n}$ corresponden con los componentes del vector descriptor $D1$, $q_{n}$ con los componentes del vector descriptor $D2$, y $n$ en el numero de componentes de ambos vectores. Por otro lado, para los descriptores binarios, se calcula la distancia \textit{Hamming}, dada por la siguiente expresión:
\begin{displaymath}
D = ||D_{1} \oplus D_{2}||
\end{displaymath}
Este proceso se repite hasta tener la distancia de cada punto de la imagen $1$ con todos los puntos de la imagen $2$, y viceversa. Al tener estas distancias, los puntos se emparejarán si y solo si se cumplen las siguientes condiciones: 

\begin{enumerate}[label=(\roman*)]
	\item El punto $P_{1}$ presenta la mejor distancia con $P_{2}$, en relación a todos los puntos de la imagen $2$.
	
	\item El punto $P_{2}$ presenta la mejor distancia con $P_{1}$ en relación a todos los puntos de la imagen $1$.
\end{enumerate}

Este proceso es mejor conocido como emparejamiento por fuerza bruta, ya que se compara entre todos los puntos por la mejor pareja posible. Si bien se asegura obtener el mejor emparejamiento, siendo viable para trabajar con pocos datos, el hecho de probar todas los casos posibles cuando se tiene una gran cantidad de puntos, incrementa en gran medida el tiempo de computo. Para efectuar este proceso de una forma eficiente se desarrollaron algoritmos basados en la búsqueda de vecinos mas cercanos. En este sentido se cuenta con el algoritmo \textit{kd-forest} (abreviado del inglés: k-dimensional forest), el cual es una mejora del algoritmo \textit{kd-tree} (abreviado del inglés: k-dimensional tree) para mejor desempeño al usar vectores multidimensionales, implementado en la librería para la rápida aproximación de vecinos mas cercanos FLANN \cite{flann} (del inglés: Fast Library for Approximate Nearest Neighbors).

%\textit{kd-tree} es una estructura de datos de segmentación de espacios para organizar puntos en un espacio de k dimensiones

\section{Módulo comparativo}

Una vez se conocen los algoritmos que se proponen utilizar, es necesario comparar su rendimiento bajo distintas condiciones, de modo que se pueda seleccionar el indicado para cada tipo de aplicación. En este sentido, se pretende estudiar el rendimiento en base a los siguientes parámetros:

\begin{itemize}
	\item \textbf{Tiempo de ejecución:} Tiempo en el que se detectan y describen las características en dos imágenes con las mismas dimensiones.
	
	\item \textbf{Cantidad de puntos detectados:} Cantidad total de puntos detectados en dos imágenes.
	
	\item \textbf{Cantidad de puntos emparejados:} Cantidad total de puntos emparejados correctamente luego de descartar parejas erróneas. 
\end{itemize}

Adicionalmente, se propone aplicar algunos métodos que permiten aumentar el rendimiento de los algoritmos de detección, así como también implementar un proceso que permita reducir o eliminar falsos positivos al emparejar puntos de interés. 

\subsection{Metodología}

El proceso planteado para la comparación, consiste en evaluar los parámetros antes mencionados para todas las combinaciones de extractores y emparejadores que se tienen, tal y como se ilustra en la figura \ref{imagen:comparacion}.

\begin{figure}[H]
	\centering
	\includegraphics[width=14cm]{comparacion.pdf}
	\caption[Combinación de algoritmos para el analisis de rendimiento]{Combinaciones posibles del módulo comparativo}
	\label{imagen:comparacion}
\end{figure}

Cuando se utiliza un algoritmo para emparejar características, es muy común que existan parejas erróneas, puesto que al tener una gran cantidad de datos, varios pares de descriptores pueden tener la similitud necesaria para ser considerados como el mismo punto. Por esta razón es importante emplear una etapa que permita filtrar dichas parejas. Como ya se mencionó, se tienen distintos tipos de emparejadores, y en función a cada uno es necesario realizar el descarte de manera distinta. A continuación se describe el proceso en función a cada caso.

En el caso del algoritmo de fuerza bruta: Se obtiene la distancia de la mejor pareja, luego se descartan todas las pareas cuya distancia sea mayor que la mejor obtenida multiplicada por un factor de umbral. El proceso planteado se muestra en el algoritmo \ref{fuerzabruta}.

\begin{figure}[h]
	\centering
	\begin{minipage}{.7\linewidth}
		\begin{algorithm}[H] %or another one check
			\caption{Selección de buenas parejas - Fuerza bruta}
			\label{fuerzabruta}
			\SetAlgoLined
			$P{i}$ $\equiv$  $i$-ésima pareja\\
			$D_{i}$ $\equiv$ Distancia de la $i$-ésima pareja\\
			$N$ $\equiv$ Número de puntos emparejados\\
			$U$ $\equiv$ Umbral para descartar erróneos\\
			\Begin{
				$U = 0.8$\;
				\ForEach{i $\in$ n=1,2,$\cdots$,N}{
					\If{$D_{i} < mejorDistancia$}{
						$mejorDistancia = Dp_{i}$\;
					}
				}
				\ForEach{i $\in$ n=1,2,$\cdots$,N}{
					\If{$D_{i} > mejor distancia \cdot U$}{
						eliminar $P_{i}$\;
					}
				}
			}
		\end{algorithm}
	\end{minipage}
\end{figure}

En el caso del algoritmo bajo el esquema de vecinos mas cercanos: Se descarta cada pareja cuya distancia esté muy cercana a la distancia del vecino mas próximo. El proceso planteado se muestra en el algoritmo \ref{vecinosmascercanos}.

\begin{figure}[h]
	\centering
	\begin{minipage}{.75\linewidth}
		\begin{algorithm}[H] %or another one check
			\caption{Selección de buenas parejas - Vecinos mas cercanos}
			\label{vecinosmascercanos}
			\SetAlgoLined
			$P{i}$ $\equiv$ $i$-ésima pareja\\
			$D_{i}$ $\equiv$ Distancia de la $i$-ésima pareja\\
			$V_{i}$ $\equiv$ Distancia del vecino mas cercano de la $i$-ésima pareja\\
			$N$ $\equiv$ Número de puntos emparejados\\
			$U$ $\equiv$ Umbral para descartar erróneos\\
			\Begin{
				$U = 0.5$\;
				\ForEach{i $\in$ n=1,2,$\cdots$,N}{
					\If{$D_{i} > V_{i} \cdot U$}{
						eliminar $P_{i}$\;
					}
				}
			}
		\end{algorithm}
	\end{minipage}
\end{figure}

En segundo lugar, con el objetivo de mejorar la detección se implementa un proceso de pre-procesamiento en las imágenes de entrada. Tal y como se estudió en la sección teórica, para que un punto característico sea detectado, éste debe contener información que lo diferencie de su entorno cercano, en este caso esta diferencia es medida en función a su intensidad. En base a esta definición, se propone un algoritmo que permite mejorar el contraste en la imagen mediante la técnica de estirar su histograma. De esta forma cada imagen tendrá el máximo rango de excursión sobre las intensidades de sus puntos, permitiendo que los valores de umbral al detectar regiones, se superen con más facilidad.

Es importante destacar que para este proceso todas las imágenes fueron almacenadas usando variables de 8 bits, es decir, que cada valor es representado en un rango comprendido entre 0 y 255.

Para el calculo del histograma se obtiene la frecuencia de aparición de cada valor de intensidad en la imagen. Luego, se obtiene el valor de la intensidad para la cual, la cantidad de píxeles cuyo valor sea menor o igual a ésta, no supere el 1\% de la cantidad total de píxeles en la imagen, este punto es llamado percentil bajo. A continuación se repite este proceso par ubicar el percentil alto, el cual corresponde con la intensidad para la cual la cantidad de píxeles inferiores a ésta, no supera el 99\% de la cantidad total de píxeles. Al tener estos valores se aplican los siguientes criterios:

\begin{itemize}
	\item Todos los píxeles cuyo valor se encuentre por debajo del percentil bajo es igualado a 0.
	\item Todos los píxeles cuyo valor se encuentre por encima del percentil alto es igualado a 255.
	\item Todos los valores cuyo valor se encuentre entre el percentil bajo y el alto, es reescalado usando la siguiente ecuación: 
	\begin{displaymath}
		I_{pixel} = \frac{(I_{pixel} - P_{bajo}) \cdot 255}{P_{alto} - P_{bajo}}
	\end{displaymath}
\end{itemize}

En la figura \ref{imagen:stretch} se puede apreciar el cambio, tanto en el histograma como en la misma imagen, luego de aplicar el proceso previamente descrito.


\begin{figure}[h]
	\centering
	\vspace{0.6cm}
	\begin{tabular}{@{}cc@{}}
		\includegraphics[width=.50\textwidth]{gray-01} &
		\includegraphics[width=.40\textwidth]{hist-01} \\
		\includegraphics[width=.50\textwidth]{gray-stretch-01} &
		\includegraphics[width=.40\textwidth]{hist-stretch-01}
	\end{tabular}
	\captionof{figure}[Estiramiento del histograma]{Arriba: imagen original y su respectivo histograma a su derecha. Abajo: Imagen luego de aplicar el estiramiento del histograma, y su respectivo histograma a su derecha.}
	\label{imagen:stretch}
\end{figure}


%%%%%%%%%%%%%%%%%%%%%%%%%%%%%%%%%%%%%%%%%%%%%%%%%%%%%%%%%%%%%%%%%%%%%%%%%%%%%%%%%%%
\subsection{Resultados}

A continuación se presentan los resultados, para todas las combinaciones de extractores y emparejadores, bajo distintas condiciones de escena. Todas las pruebas mostradas en esta sección se realizaron utilizando un equipo con las siguientes características:  \textit{Intel\textsuperscript \textregistered } Core 2 Duo CPU E8400 @ 3.00Ghz.

En el cuadro \ref{0234} se observan los resultados para 61 pares de imágenes del fondo marino en la región de Chuspa (conjunto \textit{Chuspa}), ubicada en el estado Vargas, Venezuela. Adicionalmente, en la figura \ref{imagen:0234} se ilustran cuatro imágenes representativas del conjunto estudiado. Para esta prueba se utilizaron imágenes de $640\times480$ píxeles.

\begin{table}[h]
	\centering
	\captionof{table}{Comparación de rendimiento usando imágenes del conjunto \textit{Chuspa}}
	\label{0234}
	\renewcommand{\arraystretch}{0.8}% Tighter
	\begin{tabular}{@{}lllllll@{}}
		\toprule
			 &                				& SIFT 			& SURF & ORB 			& KAZE 				& A-KAZE \\ \midrule 
		       \hfill\vline& Total parejas  & \textbf{68679}& 32888&29637			& 6818 				& 6896   \\
		Fuerza Bruta \vline& Buenas parejas & 12601			& 3764 & 224 			& 1407 				& 520    \\
			   \hfill\vline& Precisión (\%) & 18.34			&11.44 &0.70 			&20.63 				& 7.54  \\
			   \vspace{0.3cm}
			   \hfill\vline& Tiempo (s)     & 38.27			&13.60 &\textbf{2.32}	&69.09 				& 16.99  \\
			   
			   \hfill\vline& Total parejas  & \textbf{68679}& 32888&29637			& 6818 				& 6896   \\
		FLANN  \hfill\vline& Buenas parejas &\textbf{13221} & 3895 & 282 			& 1416 				&553     \\ 
			   \hfill\vline& Precisión (\%) & 19.25			& 11.84& 0.95			& \textbf{20.76}	& 8.01 \\ 
			   \hfill\vline& Tiempo (s)     & 38.27			&13.60 &\textbf{2.32} 	&69.09 				& 16.99  \\
			   \bottomrule
	\end{tabular}
\end{table}


\begin{table}[h]
	\centering
	\captionof{table}{Comparación de rendimiento usando imágenes del conjunto \textit{Chuspa}, luego de aplicar estiramiento del histograma}
	\label{0234-2}
	\renewcommand{\arraystretch}{0.8}% Tighter
	\begin{tabular}{@{}lllllll@{}}
		\toprule
		&                				& SIFT 			& SURF & ORB 			& KAZE 				& A-KAZE \\ \midrule 
		\hfill\vline& Total parejas  & \textbf{235284}  & 98402&30500			& 59061 			& 62957   \\
		Fuerza Bruta \vline& Buenas parejas & 29200		& 7461 & 200 			& 11812 			& 3115    \\
		\hfill\vline& Precisión (\%) & 12.41			&7.58 &0.65 			&19.99 				& 4.94  \\
		\vspace{0.3cm}
		\hfill\vline& Tiempo (s)     & 113.29			&28.81 &\textbf{3.44}	&75.40 				& 21.66  \\
		
		\hfill\vline& Total parejas  & \textbf{235284}  & 98402&30500			& 59061 			& 62957   \\
		FLANN \hfill\vline& Buenas parejas &\textbf{31917}& 7461 & 257 			& 12522				& 3617     \\ 
		\hfill\vline& Precisión (\%) & 13.53			& 8.17& 0.84			& \textbf{21.20}	& 5.74 \\ 
		\hfill\vline& Tiempo (s)     & 63.81			&25.79 &\textbf{3.78} 	&75.20 				& 20.58  \\
		\bottomrule
	\end{tabular}
\end{table}

\begin{figure}[h]
	
	\centering
	\vspace{0.6cm}
	\begin{tabular}{@{}cccc@{}}
		\includegraphics[width=.23\textwidth]{0234-1} &
		\includegraphics[width=.23\textwidth]{0234-2} &
		\includegraphics[width=.23\textwidth]{0234-3} &
		\includegraphics[width=.23\textwidth]{0234-4} 
	\end{tabular}
	\captionof{figure}{imágenes representativas del conjunto \textit{Chuspa}}
	\label{imagen:0234}
\end{figure}


En el cuadro \ref{SR} se observan los resultados para 42 pares de imágenes. Estas imágenes fueron proporcionadas por el centro Australiano de Robótica de Campo ACFR (del inglés: Australian Center for Field Robotics) y pertenecen al conjunto de datos \textit{ScottReef 25}. Adicionalmente, en la figura \ref{imagen:SR} se ilustran cuatro imágenes representativas del conjunto estudiado. Para esta prueba se utilizaron imágenes de $640\times480$ píxeles.

%\begin{figure}{\linewidth}
\begin{table}[h]
		\centering
		\captionof{table}{Comparación de rendimiento usando imágenes submarinas del conjunto \textit{ScottReef 25}}
		\label{SR}
		\renewcommand{\arraystretch}{0.8}% Tighter
		\begin{tabular}{@{}lllllll@{}}
			\toprule
			&              	 			& SIFT 			& SURF & ORB & KAZE & A-KAZE \\ \midrule 
			\hfill\vline& Total parejas  &\textbf{100620}&48955 &21000&17952 & 19430  \\
			Fuerza Bruta \vline& Buenas parejas & 3631 			& 1383 & 55  & 1511 & 239    \\
			\hfill\vline& Precisión (\%) & 3.60			&2.82  &0.26 & 8.41& 1.23  \\
			\vspace{0.3cm}
			\hfill\vline& Tiempo (s)     & 49.87& 16.97&\textbf{5.90} &52.74 & 15.05 \\
			
			\hfill\vline& Total parejas  &\textbf{100620}&48955 &21000			&17952 			& 19430  \\
			FLANN  \hfill\vline& Buenas parejas &\textbf{3987}  & 1480 & 65  			& 1604 			&282     \\ 
			\hfill\vline& Precisión (\%) & 3.96			&3.02  &0.30			& \textbf{8.93}& 1.45  \\
			\hfill\vline& Tiempo (s)     & 36.82			& 15.70& 5.95	& 51.48			& 15.22  \\ 
			\bottomrule
		\end{tabular}
\end{table}


\begin{table}[h]
	\centering
	\captionof{table}{Comparación de rendimiento usando imágenes submarinas del conjunto \textit{ScottReef 25}, luego de aplicar estiramiento del histograma}
	\label{SR-2}
	\renewcommand{\arraystretch}{0.8}% Tighter
	\begin{tabular}{@{}lllllll@{}}
		\toprule
		&              	 			& SIFT 			 & SURF & ORB & KAZE  & A-KAZE \\ \midrule 
		\hfill\vline& Total parejas  &\textbf{231242}&115193&21000&134065 & 127859  \\
		Fuerza Bruta \vline& Buenas parejas & 5987 	& 2317 & 62  & 7621 & 796    \\
		\hfill\vline& Precisión (\%) & 2.58			&2.01  &0.29 &5.68& 0.62  \\
		\vspace{0.3cm}
		\hfill\vline& Tiempo (s)     & 125.38& 36.427& 7.08 &78.11 & 35.47 \\
		
		\hfill\vline& Total parejas  		&\textbf{231242}		&115193 &21000	&134065 		& 127859  \\
		FLANN  \hfill\vline& Buenas parejas &\textbf{6847}  		& 2542  & 65  	& 9034 			&1052     \\ 
		\hfill\vline& Precisión (\%) 		& 2.96					&2.20   &0.30	& \textbf{6.73} & 0.82  \\
		\hfill\vline& Tiempo (s)     		& 60.10					&29.59 & \textbf{7.02}	& 66.69			& 26.72  \\ 
		\bottomrule
	\end{tabular}
\end{table}
	 
\begin{figure}[h]
 	\centering
 	\vspace{0.6cm}
 	\begin{tabular}{@{}cccc@{}}
 		\includegraphics[width=.23\textwidth]{sr1} &
 		\includegraphics[width=.23\textwidth]{sr2} &
 		\includegraphics[width=.23\textwidth]{sr3} &
 		\includegraphics[width=.23\textwidth]{sr4} 
 	\end{tabular}
 	
 	\captionof{figure}{imágenes representativas del conjunto \textit{ScottReef 25}}
 	\label{imagen:SR}
 	
\end{figure}

En el cuadro \ref{0752} se observan los resultados para 51 pares de imágenes del fondo marino del parque nacional Mochima (conjunto \textit{Mochima}), ubicado en el estado Sucre, Venezuela. Adicionalmente, en la figura \ref{imagen:0234} se ilustran cuatro imágenes representativas del conjunto estudiado. Para esta prueba se utilizaron imágenes de $640\times480$ píxeles.

%\begin{figure}{\linewidth}
\begin{table}[h]
	\centering
	\captionof{table}{Comparación de rendimiento usando imágenes del conjunto \textit{Mochima}}
	\label{0752}
	\renewcommand{\arraystretch}{0.8}% Tighter
	\begin{tabular}{@{}lllllll@{}}
		\toprule
		&                      				& SIFT 			& SURF & ORB & KAZE & A-KAZE  \\ \midrule 
		\hfill\vline& Total parejas  &\textbf{116840}& 62472&25382&35934 & 34733   \\
		Fuerza Bruta \vline& Buenas parejas & 14659			& 6068 & 207 & 6760 & 1888    \\
		\hfill\vline& Precisión (\%) & 12.58			&9.71  &0.81 & 18.81 & 5.43  \\
		\vspace{0.3cm}
		\hfill\vline& Tiempo (s)     & 53.24			&18.37 & \textbf{2.30} &61.14 & 15.37   \\
		
		\hfill\vline& Total parejas  &\textbf{116840}& 62472&25382			&35934 				& 34733   \\
		FLANN  \hfill\vline& Buenas parejas &\textbf{15690} & 6426 & 242 			& 7175 				& 2136    \\
		\hfill\vline& Precisión (\%) & 13.42			& 5.48 &0.95  			& \textbf{19.96} 	& 6.14    \\ 
		\hfill\vline& Tiempo (s)     & 39.23			& 17.66& 2.44	& 62.07				& 15.07   \\ 
		\bottomrule
	\end{tabular}
\end{table}

\begin{table}[h]
	\centering
	\captionof{table}{Comparación de rendimiento usando imágenes del conjunto \textit{Mochima}, luego de aplicar estiramiento del histograma}
	\label{0752-2}
	\renewcommand{\arraystretch}{0.8}% Tighter
	\begin{tabular}{@{}lllllll@{}}
		\toprule
		&                      				& SIFT 			& SURF & ORB 		& KAZE 				& A-KAZE  \\ \midrule 
		\hfill\vline& Total parejas  &\textbf{146453}		& 75342&25440		&61107 				& 57797   \\
		Fuerza Bruta \vline& Buenas parejas & 16743			& 6879 & 208 		& 12105 			& 3310    \\
		\hfill\vline& Precisión (\%) & 11.43				&3.13  &0.81		& 19.80 			& 5.72  \\
		\vspace{0.3cm}
		\hfill\vline& Tiempo (s)     & 68.42				&22.05 & \textbf{2.80} &66.45       & 18.69   \\
		
		\hfill\vline& Total parejas  &\textbf{146453}& 75342&25440			&61107 				& 57797   \\
		FLANN  \hfill\vline& Buenas parejas &\textbf{18006} & 7328 & 244 			& 12890 			& 3773    \\
		\hfill\vline& Precisión (\%) & 12.29				& 9.72 &0.95  			& \textbf{21.09} 	& 6.52    \\ 
		\hfill\vline& Tiempo (s)     & 45.01				& 20.27& 2.88			& 64.77				& 17.57   \\ 
		\bottomrule
	\end{tabular}
\end{table}

\begin{figure}[h]
	\centering
	\vspace{0.6cm}
	\begin{tabular}{@{}cccc@{}}
		\includegraphics[width=.23\textwidth]{0752-1} &
		\includegraphics[width=.23\textwidth]{0752-2} &
		\includegraphics[width=.23\textwidth]{0752-3} &
		\includegraphics[width=.23\textwidth]{0752-4} 
	\end{tabular}
	\captionof{figure}{imágenes representativas del conjunto \textit{Mochima}}
	\label{imagen:0752}
\end{figure}

En el cuadro \ref{geotagg} se observan los resultados para 50 pares de imágenes aéreas de una cantera de grava (conjunto \textit{Grava}), ubicado en Suiza. Estas imágenes fueron proporcionadas por la empresa SenseFly\footnote{\url{https://www.sensefly.com/education/datasets/}} y capturadas por un dron eBee\footnote{\url{https://www.sensefly.com/drone/ebee-mapping-drone/}}.  Adicionalmente, en la figura \ref{imagen:geotag} se ilustran cuatro imágenes representativas del conjunto estudiado. Para esta prueba se utilizaron imágenes de $1280\times960$ píxeles.

\begin{table}[h]
	\centering
	\caption{Comparación de rendimiento usando imágenes del conjunto \textit{Grava}}
	\label{geotagg}
	\begin{tabular}{@{}lllllll@{}}
		\toprule
		&                      				& SIFT 			& SURF & ORB & KAZE  & A-KAZE  \\ \midrule 
			   \hfill\vline& Total parejas  &\textbf{336189}& 222251&25000&108819& 121914   \\
		Fuerza Bruta \vline& Buenas parejas & 21006			& 5853 & 126 & 10290 & 3240  \\
			   \hfill\vline& Precisión (\%) & 3.24			&2.63  & 0.5   & 9.45& 2.65  \\
				\vspace{0.3cm}
			   \hfill\vline& Tiempo (s)     & 282.76		&116.38 & 28.84 &309.29 & 92.71   \\
		
			   \hfill\vline& Total parejas  &\textbf{336189}& 222251&25000			&108819				& 121914   \\
		FLANN  \hfill\vline& Buenas parejas &\textbf{22610} & 6525 & 143			& 10861				& 3749    \\
			   \hfill\vline& Precisión (\%) & 6.72			& 2.93 &0.57  			& \textbf{9.98} 	& 3.07    \\ 
			   \hfill\vline& Tiempo (s)     & 153.12		& 90.27& \textbf{28.53}	& 299.59			& 81.52   \\ 
		\bottomrule
	\end{tabular}
\end{table}
\begin{table}[h]
	\centering
	\caption{Comparación de rendimiento usando imágenes del conjunto \textit{Grava}, luego de aplicar estiramiento del histograma}
	\label{geotagg-2}
	\begin{tabular}{@{}lllllll@{}}
		\toprule
		&                      				& SIFT 			& SURF & ORB & KAZE  & A-KAZE  \\ \midrule 
		\hfill\vline& Total parejas  &\textbf{335418}		& 221287&25000&108124& 121122   \\
		Fuerza Bruta \vline& Buenas parejas & 20943			& 5812 & 127 & 10313 & 3239  \\
		\hfill\vline& Precisión (\%) & 6.24			&2.62  & 0.58  & 9.53& 2.67  \\
		\vspace{0.3cm}
		\hfill\vline& Tiempo (s)     & 278.97		&114.86 & 29.54 &300.12 & 92.31   \\
		
		\hfill\vline& Total parejas  &\textbf{335418}& 221287&25000			&108124				& 121122   \\
		FLANN  \hfill\vline& Buenas parejas &\textbf{22555} & 6509 & 149	& 10880				& 3754    \\
		\hfill\vline& Precisión (\%) & 6.72			& 2.94 &0.59  			& \textbf{10.06} 	& 3.09   \\ 
		\hfill\vline& Tiempo (s)     & 153.71		& 88.25& \textbf{29.44}	& 294.61			& 83.68   \\ 
		\bottomrule
	\end{tabular}
\end{table}

\begin{figure}[h]
	\centering
	\begin{tabular}{@{}cccc@{}}
		\includegraphics[width=.23\textwidth]{geotagg-1} &
		\includegraphics[width=.23\textwidth]{geotagg-2} &
		\includegraphics[width=.23\textwidth]{geotagg-3} &
		\includegraphics[width=.23\textwidth]{geotagg-4} 
	\end{tabular}
	\caption{imágenes representativas del conjunto \textit{Grava}}
	\label{imagen:geotag}
\end{figure}

Presentadas las pruebas, analizaremos el rendimiento de todas las combinaciones estudiadas en función a los parámetros descritos al inicio de la sección.

Es evidente la inversa relación entre la cantidad de puntos detectados y el tiempo que le toma ese proceso. En función a la detección, podemos acotar que los algoritmos SIFT y KAZE presentan la mayor cantidad de parejas correctas, siendo SIFT el que permanece con el mayor numero a lo largo de todas las pruebas. Este resultado está relacionado con el uso de descriptores vectoriales, que permiten una descripción mas robusta de los puntos detectados. Además que en el caso de SIFT no se realiza ninguna aproximación para la obtención de las diferentes escalas de la imagen.

Por el contrario, Aquellos algoritmos que utilizan descriptores binarios presentan un menor rendimiento en base a parejas emparejadas. Sin embargo, esta característica les permite realizar la extracción con una velocidad mucho mayor, siendo este un patrón que se mantiene a lo largo de todas las pruebas.

Analizando los resultados del extractor \textit{KAZE}, Si bien no ofrece la mayor cantidad de parejas correctas, presenta el mejor cociente entre parejas totales y correctas. Lo que indica un gran nivel de robustez en su descriptor, ya que un alto porcentaje de las parejas encontradas en un principio, efectivamente corresponden con el mismo punto.

Comparando el rendimiento de los emparejadores, podemos observar que para la mayoría de los casos el algoritmo FLANN aumenta la cantidad de parejas correctas, y al mismo tiempo disminuye el tiempo computacional. Si bien para la mayoría de las pruebas no se tiene un mejora significativa en términos de tiempo, es debido a que este algoritmo ofrece mayores beneficios cuando la cantidad de parejas a estudiar es mucho mas elevada. Este punto se evidencia mayormente en los resultados del algoritmo SIFT, el cual ofrece mayor cantidad de parejas, y por lo tanto mayor conjunto de puntos a emparejar.

Por ultimo podemos destacar el importante aumento en la cantidad de parejas detectadas, para los casos en los que se aplicó el pre-procesamiento en la entrada. Como ya se mencionó, se debe al aumento en el contraste, lo que produce que la diferencia entre un punto característico y su vecindario se incremente, aumentando al mismo tiempo la posibilidad de ser detectado y luego emparejado.

% Intel Core 2 Duo CPU E8400 @ 3.00 Ghz
% 3.7 GiB RAM

\section{Resumen}

El proceso de establecer la relación entre las imágenes, es un paso muy importante en la etapa de registro. De esta forma, la selección del algoritmo para la extracción de características, así como también la técnica que se use para establecer las correspondencias determinará la calidad final del mapa. Tal y como se presentó, se dispone de una gran cantidad de algoritmos para la extracción de características locales, así como también métodos eficientes para emparejarlos. Esta diversidad permite que se cuente con algoritmos que se comporten de manera eficiente para cada tipo de aplicación, o en el caso de mapeo, para cada tipo de escena. 

En el presente capítulo se logró establecer una clasificación sobre los algoritmos de generación, en función a como aborden las etapas mas importantes de éste. Tomando en cuenta los distintos métodos y técnicas, se planteó un sistema que combina aquellas que presentan los mejores resultados, considerando los requerimientos del presente proyecto. Al mismo tiempo se describió el funcionamiento de todos algoritmos que se plantean implementar en el sistema, permitiendo que se puedan plantear técnicas para mejorar el rendimiento en la extracción y emparejamiento. 

Tras los resultados mostrados, se logró evidenciar claramente el rendimiento de cada combinación en función a los parámetros de tiempo y cantidad de parejas. Lo que permite una selección en función de los requerimientos de la aplicación, y debido a la modularidad del esquema planteado es posible reemplazar cualquiera de los algoritmo presentes para extraer y emparejar sin afectar el resto del proceso para la construcción del mosaico.
\chapter{Algoritmos para la generacion de mosaico}
\label{capitulo4}
\lhead{Capítulo 4. \emph{Algoritmos para la generacion de mosaico}}

Resumen del capitulo 4

\section{Seccion 1}
mensaje de prueba

\section{Sección 2}
\subsection{sub-sección 2}
mensaje de prueba subsección 2



\chapter{Unión de imágenes}
\label{capitulo5}
\lhead{Capítulo 5. \emph{Unión de imágenes}}


\section{Introducción}
El objetivo final de la creación de un mosaico, es lograr un mapa que represente de la mejor forma la trayectoria recorrida. Esto es, que sea visualmente congruente y que no presente ningún tipo de discontinuidades de modo que todo parezca una misma imagen. De las primeras etapas se manifiestan muchos errores producto de los problemas ya planteados, si bien a lo largo del proceso éstos se intentan reducir lo mas posible, siempre es necesaria la etapa final de fusión para lograr los objetivos propuestos.

En esta sección se describen los algoritmos utilizados para lograr fusionar las imágenes del mosaico, como ya se especificó fueron seleccionados aquellos que presentaron resultados importantes en diversos estudios externos, y se implementaron en conjunto para lograr resultados mucho mas robustos. Estos se detallan a continuación en el orden de aplicación sobre el mosaico: linea de corte, ajuste de color y finalmente la fusión ponderada.

Finalmente se muestran los resultados con su respectivo análisis de aplicar cada uno de los algoritmos aquí descritos.
\clearpage


\section{Linea de costura}
Este algoritmo es aplicado en primer lugar, ya que el resto de métodos requieren conocer de antemano los límites de cada imagen. 
A diferencia de los anteriores, este tipo de algoritmos es el único que toma en cuenta la información que comparten las imágenes en el área que tienen en común, con lo cual su implementación logra corregir la mayor cantidad de imperfecciones. 


\subsection{Corte por grafo}

Este es un método derivado de la teoría de grafos, donde la idea es separar un grafo con conexiones simples en dos grafos separados con un mínimo costo de separación. Se define un grafo $\mathcal{G} = \langle \mathcal{N}, \mathcal{E} \rangle$ como un conjunto de nodos $\mathcal{N}$, conectados por enlaces $\mathcal{E}$. Cada enlace conecta dos nodos y tiene asociado un costo o peso $\mathcal{W}(p, q) \,\, p,q \in \mathcal{N}$ --- cuando se habla de conexión simple, se refiere a que el costo en los enlaces está asociado para ambas direcciones $\mathcal{W}(p, q) = \mathcal{W}(p, q)$ ---. Se dice que dos grafos están separados si no se tiene ningún enlace que conecte dos nodos entre los grafos. Definimos $\mathcal{S}$ y $\mathcal{T}$ como los grafos separados que se tienen luego de aplicar el corte en $\mathcal{G}$. El método para determinar el mejor corte se basa en encontrar el camino entre los enlaces que logra separa un grafo en dos, con el mínimo costo de corte, donde el costo del corte es la suma de los pesos de todos los enlaces del camino seleccionado.

En el conjunto de nodos en el grafo se cuentan con dos especiales o nodos terminales, el inicio ($\mathtt{I}$) y el final ($\mathtt{F}$), donde el resto de píxeles en la imagen corresponden con un nodo no terminal. En aplicaciones de visión por computadora, cuando se desea unir dos imágenes en una región de intersección, lo que se busca es lograr un etiquetado de píxeles que permita distinguir que nodos corresponden a cada imagen en el área de intersección.

Se presenta en la ecuación \ref{funcion-corte} la función de coste que etiqueta los nodos, y minimiza el costo del corte.
\begin{equation}
C(f) = \sum_{_p\in \mathcal{N}}^{} D_p(f_p) + \sum_{_p,_q \in \mathcal{N} - \{\mathtt{I},\mathtt{F}\}}^{} \mathcal{W}_p,_q (f_p, f_q)
\label{funcion-corte}
\end{equation}
Donde $p$ es un nodo que pertenece al conjunto de nodos no terminales $\mathcal{N} - \{\mathtt{I},\mathtt{F}\}$. El término $D_p(f_p)$ es el costo de asignar una etiqueta $f_p$ ($f_p \in \{0,\,1\}$) al nodo $p$ --- en este caso una etiqueta binaria, asociando el píxel a una imagen u otra ---. El término $\mathcal{W}_p,_q (f_p, f_q)$ es el costo de asociar una etiqueta al nodo $_p$ y una distinta al nodo $_q$.

Una gran diferencia de intensidades entre píxeles adyacentes representa un fuerte indicador de la existencia de un borde o contorno entre dos objetos, es decir, que el costo de un enlace se puede definir como el inverso de la diferencia entre la intensidad de los píxeles que conecta. Siendo $I(p)$ la intensidad de un píxel $p$, se defina el coso de cada enlace como:
\begin{equation}
\mathcal{W}_p,_q = 255 - |I(p) - I (q)|
\label{costo-corte}
\end{equation}
Si bien se consideran los parámetros necesarios con esa función, se obtienen resultados mucho mas robustos usando una función exponencial \cite{graph-opencv}:
\begin{equation}
\mathcal{W}_p,_q = e^{\left(\frac{255-|I(p) - I (q)|}{2 \sigma}  \right) }
\label{costo-corte}
\end{equation}
Donde $\sigma$ es la desviación estándar de la imagen, y siendo válido para los casos en los que $f_p \neq f_q$, indicando que ambos nodos pertenecen serán separados por la linea de corte.

\begin{figure}[h]
	\centering
	\includegraphics[width=1\linewidth]{grafo-completo.pdf}
	\caption[Corte por grafo]{De izquierda a derecha: imagen original, creación de nodos y enlaces, se asignan los pesos y se halla la linea de corte, finalmente se binariza la imagen según las etiquetas para crear una mascara.}
	\label{imagen:grafo}
\end{figure}
Refiriéndonos a la figura \ref{imagen:grafo}, se observa el proceso de modelar una imagen mediante un grafo con conexiónes simples, donde cada cuadro está compuesto por el inverso de la diferencia de intensidades entre dos imágenes, representado en escala de grises. Al final se tiene un imagen compuesta por dos grafos separados, donde $\mathtt{I} \in \mathcal{S}$ y $\mathtt{F} \in \mathcal{T}$ donde a cada nodo del grafo se le asigna un valor binario dependiendo de la etiqueta resultante el algoritmo de minimización del costo de corte.


\section{Corrección de color}
Cuando se tienen imágenes capturadas desde distintos puntos de vista, se suelen tener en la composición final cambios de intensidades por cambios de exposición de luz en la escena. Por ellos es necesario 

\subsection{Método de Reinhard}
Prueba

\section{Fusión de imágenes}
prueba

\subsection{Fusión ponderada}
Prueba

\subsection{Fusión ponderada piramidal}
Prueba

\section{Resultados}

\section{Resumen}



\chapter{Conclusiones y trabajos futuros}
\label{capitulo6}
\lhead{Capítulo 6. \emph{Conclusiones y trabajos futuros}}

Conclusiones

	
% El estilo de la bibliografía es AAAI, definido en el archivo aaai.bst.
\label{Bibliography}
\bibliographystyle{unsrt}
\bibliography{bibliografia}
\lhead{\emph{Bibliografía}}
\addtocontents{toc}{\vspace{2em}}
	
% Apéndices
\appendix
\chapter{Instalación de librería OpenCV y dependencias}
\label{apendiceA}
\lhead{Apéndice A. \emph{Instalación de librería OpenCV y dependencias}}

% En los apéndices se incluye cualquier información que no sea esencial para la
% comprensión básica del trabajo, pero provea ejemplos y casos de estudio
% extendidos que permitan un análisis más exhaustivo.

\section{Instalación en el SO Linux bajo la distribución Ubuntu 16.04}

\section*{Dependencias}

\subsection*{Esenciales}

\begin{lstlisting}[language=bash]
$ sudo apt-get install build-essential cmake pkg-config
\end{lstlisting}

\subsection*{Codificador de imágenes}

\begin{lstlisting}[language=bash]
$ sudo apt-get install libjpeg8-dev libtiff5-dev 
libjasper-dev libpng12-dev
\end{lstlisting}

\subsection*{Codificador de vídeos}

\begin{lstlisting}[language=bash]
$ sudo apt-get install libavcodec-dev libavformat-dev 
libswscale-dev libv4l-dev
$ sudo apt-get install libxvidcore-dev libx264-dev
\end{lstlisting}

\subsection*{Interfaz de usuario con GTK}

\begin{lstlisting}[language=bash]
$ sudo apt-get install libgtk-3-dev
\end{lstlisting}

\subsection*{Para operaciones con matrices}

\begin{lstlisting}[language=bash]
$ sudo apt-get install libatlas-base-dev gfortran
\end{lstlisting}

\section*{OpenCV 3.2}

\subsection*{Seleccionar una ruta para la instalación}

\begin{lstlisting}[language=bash]
$ cd <ruta_elegida>
\end{lstlisting}

\subsection*{Descargar el código fuente de OpenCV desde el repositorio oficial}

\begin{lstlisting}[language=bash]
$ wget https://github.com/opencv/opencv/archive/3.2.0.-
tar.gz .
$ tar xvf opencv-3.2.0.tar.gz 
\end{lstlisting}

\subsection*{Descargar las librerías de los contribuidores}

\begin{lstlisting}[language=bash]
$ wget https://codeload.github.com/opencv/opencv_con-
trib/zip/master -O opencv_contrib-master.zip
$ unzip opencv_contrib-master.zip
\end{lstlisting}

\subsection*{Soporte para OpenCL}

\begin{lstlisting}[language=bash]
$ sudo apt-get install libgtkglext1 libgtkglext1-dev
\end{lstlisting}
\newpage
\subsection*{Instalar}

\begin{lstlisting}
$ cd opencv-3.2.0/
$ mkdir build
$ cd build
	
$ cmake -D CMAKE_BUILD_TYPE=RELEASE \
  -D CMAKE_INSTALL_PREFIX=/usr/local \
  -D WITH_OPENGL=ON \
  -D WITH_CUDA=OFF \
  -D ENABLE_FAST_MATH=1 \
  -D CUDA_FAST_MATH=0 \
  -D WITH_CUBLAS=1 \
  -D INSTALL_PYTHON_EXAMPLES=OFF \
  -D OPENCV_EXTRA_MODULES_PATH=../../opencv_contrib-
     master/modules \
  -DBUILD_PNG=ON \
  -DBUILD_TIFF=ON \
  -DBUILD_JPEG=OFF \
  -DBUILD_ZLIB=ON \
  -DWITH_OPENCL=ON \
  -DWITH_OPENMP=ON \
  -DWITH_FFMPEG=ON \
  -DWITH_GTK=ON \
  -DWITH_VTK=ON \
  -DWITH_TBB=ON \
  -DINSTALL_C_EXAMPLES=ON \
  -DINSTALL_TESTS=OFF \
  -D BUILD_DOCS=OFF \
  -D OPENCV_ENABLE_NONFREE=ON \
  -D BUILD_EXAMPLES=OFF ..
	
$ make
$ sudo make install
\end{lstlisting} 



\addtocontents{toc}{\vspace{2em}}
	
\backmatter
	
\end{document}
